
\subsection{algebraische Zahlen}
    \label{subsec:algebraic-numbers}
    \textrm{Eine komplexe Zahl $z \in \mathbb{C}$ heißt \emph{algebraisch} genau dann, wenn gilt:}
    \begin{equation}
        \exists f \in \mathbb{Q}[x] : f(z) = 0 \label{eq:def-algebraic}
    \end{equation}
    \textrm{d.h.\ falls z eine Lösung eines Polynoms mit rationalen Koeffizienten $a_k~=~\frac{p_k}{q_k}$ mit $p_k \in
    \mathbb{Z},\ q_k \in \mathbb{N}\ (\forall \ 1 \leq k \leq n)$ ist.
    \newline
    Die Zahl $n = \deg f$ heißt \emph{Grad} der algebraischen Zahl $z$.
    \newline
    OBdA kann angenommen werden, dass $f \in \mathbb{Z}[x]$, da sich die Gleichung $f(x) = 0$ mit dem Produkt der Nenner
    der Koeffizienten multiplizieren lässt, wodurch alle Koeffizienten ganzzahlig werden, die Gleichung und damit auch
    das resultierende Polynom jedoch dieselben Lösungen bzw.\ Nullstellen besitzen.\ Weiterhin lässt sich oBdA annehmen,
    dass für den Koeffizienten der höchsten Potenz von $x$ gilt: $a_n \neq 0$.}

\subsection{Der Satz von Roth (\emph{Theorem I})}
    \label{subsec:th1}
    Sei $\xi \in \mathbb{R}$ algebraisch irrational und $\delta > 0$ beliebig.\ Dann besitzt die Ungleichung
    \begin{equation}
        0 < \left| \xi - \frac{p}{q} \right| < q^{-(2+\delta)} \label{eq:svr}
    \end{equation}
    \textrm{nur endlich viele Lösungen in $p \in \mathbb{Z}$, $q \in \mathbb{N}$.
    \newpage
    \textrm{Hiermit liefert der Satz die obere Schranke $2$ für das Irrationalitätsmaß algebraisch irrationaler Zahlen;
    zusammen mit der unteren Schranke von ebenfalls $2$ gilt somit $\mu(x) = 2$ für alle algebraisch irrationalen Zahlen
    $x$.}
    \newline \newline
    Zunächst wird gezeigt, dass der Satz nur für $a_n = 1$ aus~\ref{subsec:algebraic-numbers} zu zeigen ist.}

\subsection{Normieren des Polynoms einer algebraischen Zahl}
    \label{subsec:norm-poly}
    \textrm{Angenommen, der \hyperref[eq:svr]{\emph{Satz von Roth}} gelte.\ Dann folgt aus $f(\xi) = 0$ aus~\ref
    {subsec:algebraic-numbers} durch Multiplikation mit $a_n^{n-1}$ für $a_n \xi \coloneq \Xi$:}
    \begin{equation*}
        0 = \Xi^n + a_{n-1} \Xi^{n-1} + \dsum{a_{n-2}a_n \Xi^{n-2}}{a_n^{n-1} a_0}
    \end{equation*}
    \textrm{und nach Multiplikation mit $\left| a_n \right|$ und geeigneter Abschätzung von~\eqref{eq:svr} gilt:}
    \begin{equation*}
        \left| \Xi - a_n \frac{p}{q} \right| < \left| a_n \right| q^{-(2+\delta)} < q^{-(2+\den{2}\delta)}
    \end{equation*}
    \textrm{für hinreichend große $q$.\ Da $\delta$ beliebig gewählt wurde, gilt der Satz somit nun für $\Xi$ genau dann,
    wenn er für $\xi$ gilt. Somit gilt insgesamt oBdA:}
    \begin{equation}
        f(x) = x^n + \dsum{a_{n-1} x^{n-1}}{a_0} \sptext{mit} f(\xi) = 0 \sptext{und} \enum{a_{n-1}}{a_0}
        \in \mathbb{Z}. \label{eq:prereq}
    \end{equation}
    \newline
    \textrm{Sei im Übrigen}
    \begin{equation}
        a = \max\{ 1, \enum{|a_{n-1}|}{|a_0|} \}. \label{eq:def-a}
    \end{equation}
    \newline
    Dies wird im weiteren Verlauf der Arbeit und des Beweises verwendet.
    
