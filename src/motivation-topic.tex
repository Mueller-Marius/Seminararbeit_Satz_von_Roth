
\section{Motivation des Themas}
    \label{sec:motivation}
    \textrm{Das sogenannte \emph{Irrationalitätsmaß} quantifiziert die Irrationalität einer reellen Zahl. Dazu wird
    die folgende Definition verwendet:}
    
    \subsection{\textrm{Das Irrationalitätsmaß}}
        \label{subsec:irr-measure}
        \textrm{Sei $x \in \mathbb{R}$ beliebig. Sei $M$ die Menge aller $\mu \in \mathbb{R}$, sodass die Ungleichung
            \begin{equation*}
                0 < \left| x - \frac{p}{q} \right| < \den{q^\mu}
            \end{equation*}
            nur endlich viele Lösungen in $p \in \mathbb{Z}, q \in \mathbb{N}$
            besitzt.\ Dann heißt
            \begin{equation*}
                \mu(x) \coloneqq \inf(M)
            \end{equation*}
            das \emph{Irrationalitätsmaß} von $x$.
            \newline
            Die folgenden Beispiele illustrieren diese Definition.}
    
    \subsection{\textrm{Beispiele zum Irrationalitätsmaß}}
        \label{subsec:examples-irr-measure}
        \begin{itemize}
            \item \textrm{Für $x \in \mathbb{Q}$ gilt: $\mu(x) = 1$}
            \item \textrm{Für irrationale $x$ wurde gezeigt, dass gilt: $\mu(x) \geq 2$}
            \item \textrm{Für die \emph{eulersche Zahl} $e$ gilt: $\mu(e) = 2$}
            \item \textrm{Das Irrationalitätsmaß der Kreiszahl $\pi$ ist bisher unbekannt.\ Der neuste Fortschritt
            setzt die obere Schranke bei $\mu(\pi) \leq 7,1032\dots$ fest.}
        \end{itemize}
    
    \subsection{Grundlegende Aussage des \emph{Satzes von Roth}}
        \label{subsec:basically-svr}
        \textrm{Es stellt sich nach den oben genannten Beispielen die Frage, ob auch alle irrationale Zahlen dasselbe
        Irrationalitätsmaß besitzen. Hier liefert der \emph{Satz von Roth} eine teilweise Antwort:
        \newline
        Das Irrationalitätsmaß aller \emph{algebraisch} irrationalen Zahlen ist genau zwei.}
    
