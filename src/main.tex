%! Author = gurki
%! Date = 08.08.22

% Preamble
\documentclass[11pt]{article}

\title{Seminararbeit zum Vortrag ``Satz von Roth I'' im Seminar ``Analysis'' bei Prof. Dr. Hein}
\author{Marius Müller}
\date{Juli 2022}


% Packages
\usepackage{amsmath}
\usepackage{amsfonts}
\usepackage{mathtools}
\usepackage{hyperref}
\usepackage{amssymb}

\setlength{\parindent}{0pt}

\DeclareMathOperator{\ind}{ind}

\newcommand{\content}[1]{\begin{tabular}{|c|}\hline \!#1\! \end{tabular}} %for Polynomials
\newcommand{\enum}[2]{#1, \dots, #2}
\newcommand{\dsum}[2]{#1 + \dots + #2}
\newcommand{\spdots}[2]{#1 \dots #2}
\newcommand{\sptext}[1]{\ \textrm{#1}\ }
\newcommand{\den}[1]{\frac{1}{#1}}
\newcommand{\cbrt}[1]{\sqrt[3]{#1}}

\newenvironment{proof}
    {\newline \emph{Beweis}. \newline}
    {\hfill $Q.e.d.$}
\newenvironment{proofname}[1]
    {\newline \emph{Beweis}(#1): \newline}
    {\hfill $Q.e.d.$}
    


% Document
\begin{document}

    \maketitle

    \vfill

    \begin{abstract}
        \noindent \textrm{Diese Arbeit behandelt den \emph{Satz von Thue-Siegel-Roth}, der mittels des Irrationalitätsmaßes
        eine Aussage über die Irrationalität algebraisch irrationaler Zahlen liefert.
        \newline
        In dieser Arbeit wird zur Thematik hingeführt, die nötigen Grundlagen behandelt und das erste Theorem im Beweis
        des Satzes erklärt und bewiesen.}
    \end{abstract}

    \newpage

    \tableofcontents

    \newpage

    \section{Einleitung}
    \label{sec:intro}
        \textrm{Der \emph{Satz von Thue-Siegel-Roth} (im Folgenden kurz \emph{Satz von Roth} genannt) wurde erstmals von
        \emph{Klaus Friedrich Roth} bewiesen, der im Jahre 1958 für diesen Meilenstein die \emph{Fields-Medallie}
        verliehen bekam.
        \newline
        Diese Arbeit ist eng an das Kapitel VI des Buches ``An Introduction To Diophantine Approximation'' von
        John W. S. Cassels aus 1957 angelehnt.
        \newline
        Der Beweis des \emph{Satzes von Roth} gliedert sich hier in drei Theorems.\ Von diesen wird in dieser
        Arbeit der erste Satz, das \emph{Theorem II}, beschrieben, erklärt und bewiesen (der \emph{Satz von Roth} selbst
        ist hier das \emph{Theorem I}; der Übersichtlichkeit halber wird sich an die Nummerierung der Quelle gehalten;
        die Notation wurde jedoch stellenweise abgeändert.)}

    
\section{Motivation des Themas}
    \label{sec:motivation}
    \textrm{Das sogenannte \emph{Irrationalitätsmaß} quantifiziert die Irrationalität einer reellen Zahl. Dazu wird
    die folgende Definition verwendet:}
    
    \subsection{\textrm{Das Irrationalitätsmaß}}
        \label{subsec:irr-measure}
        \textrm{Sei $x \in \mathbb{R}$ beliebig. Sei $M$ die Menge aller $\mu \in \mathbb{R}$, sodass die Ungleichung
            \begin{equation*}
                0 < \left| x - \frac{p}{q} \right| < \den{q^\mu}
            \end{equation*}
            nur endlich viele Lösungen in $p \in \mathbb{Z}, q \in \mathbb{N}$
            besitzt.\ Dann heißt
            \begin{equation*}
                \mu(x) \coloneqq \inf(M)
            \end{equation*}
            das \emph{Irrationalitätsmaß} von $x$.
            \newline
            Die folgenden Beispiele illustrieren diese Definition.}
    
    \subsection{\textrm{Beispiele zum Irrationalitätsmaß}}
        \label{subsec:examples-irr-measure}
        \begin{itemize}
            \item \textrm{Für $x \in \mathbb{Q}$ gilt: $\mu(x) = 1$}
            \item \textrm{Für irrationale $x$ wurde gezeigt, dass gilt: $\mu(x) \geq 2$}
            \item \textrm{Für die \emph{eulersche Zahl} $e$ gilt: $\mu(e) = 2$}
            \item \textrm{Das Irrationalitätsmaß der Kreiszahl $\pi$ ist bisher unbekannt.\ Der neuste Fortschritt
            setzt die obere Schranke bei $\mu(\pi) \leq 7,1032\dots$ fest.}
        \end{itemize}
    
    \subsection{Grundlegende Aussage des \emph{Satzes von Roth}}
        \label{subsec:basically-svr}
        \textrm{Es stellt sich nach den oben genannten Beispielen die Frage, ob alle irrationalen Zahlen dasselbe
        Irrationalitätsmaß besitzen.
        \newline
        Dies ist zwar im Allgemeinen nicht der Fall, der \emph{Satz von Roth} liefert jedoch eine teilweise Antwort:
        \newline
        Das Irrationalitätsmaß aller \emph{algebraisch} irrationalen Zahlen ist genau zwei.}
    

    
    
\section{Grundlagen und Voraussetzungen}
    \label{sec:preliminaries}
    \textrm{In diesem Kapitel werden die Grundlagen behandelt, die für den Beweis des \emph{Satzes von Roth} benötigt
    werden. Außerdem wird die formale Aussage des Satzes dargelegt; der Beweis wird jedoch nicht in dieser Arbeit vollzogen.}
    
    \subsection{Notation}
    \label{subsec:notation}
        \textrm{In dieser Arbeit wird $\mathbb{N}$ verwendet als natürliche Zahlen \emph{ohne} Null, wogegen $\mathbb{N}_0$
        die natürlichen Zahlen \emph{inklusive} der Null meint.}
    
    
\section{Grundlagen und Voraussetzungen}
    \label{sec:preliminaries}
    \textrm{In diesem Kapitel werden die nötigen Grundlagen behandelt, die für den Beweis des \emph{Satzes von Roth}
    benötigt werden. Außerdem wird die formale Aussage des Satzes dargelegt; der Beweis wird jedoch nicht in dieser
    Arbeit vollzogen.}
    
    \subsection{Notation}
    \label{subsec:notation}
        \textrm{In dieser Arbeit wird $\mathbb{N}$ verwendet als natürliche Zahlen \emph{ohne} Null, wogegen $\mathbb{N}_0$
        die natürlichen Zahlen \emph{inklusive} der Null meint.}
    
    \subsection{algebraische Zahlen}
        \label{subsec:algebraic-numbers}
        \textrm{Eine komplexe Zahl $z \in \mathbb{C}$ heißt \emph{algebraisch} genau dann, wenn gilt:}
        \begin{equation}
            \exists f \in \mathbb{Q}[x] : f(z) = 0 \label{eq:def-algebraic}
        \end{equation}
        \textrm{d.h.\ falls z eine Lösung eines Polynoms mit rationalen Koeffizienten ist.\ Sei $n = \deg f$. OBdA kann
        angenommen werden, dass $f \in \mathbb{Z}[x]$, da sich die Gleichung $f(x) = 0$ mit dem Produkt der Nenner der
        Koeffizienten der Form $a_k = \frac{p_k}{q_k} \ (\forall \  1 \leq k \leq n)$, d.h.\ mit $\prod_{k=1}^n q_k$
        multiplizieren lässt, wodurch alle Koeffizienten ganzzahlig werden, die Gleichung und damit auch das
        resultierende Polynom jedoch dieselben Lösungen bzw.\ Nullstellen besitzen.
        \newline
        Weiterhin lässt sich oBdA annehmen, dass für den Koeffizienten der höchsten Potenz von $x$ gilt: $a_n \neq 0$.}
    
    \subsection{Satz von Roth (\emph{Theorem I})}
        \label{subsec:th1}
        Sei $\xi \in \mathbb{R}$ algebraisch irrational und $\delta > 0$ beliebig.\ Dann besitzt die Ungleichung
        \begin{equation}
            0 < \left| \xi - \frac{p}{q} \right| < q^{-(2+\delta)} \label{eq:svr}
        \end{equation}
        \textrm{nur endlich viele Lösungen in $p \in \mathbb{Z}$, $q \in \mathbb{N}$.
        \newpage
        \textrm{Hiermit liefert der Satz die obere Schranke $2$ für das Irrationalitätsmaß algebraisch irrationaler
        Zahlen; zusammen mit der unteren Schranke von ebenfalls $2$ gilt somit $\mu(x) = 2$ für alle algebraisch
        irrationalen Zahlen $x$.}
        \newline \newline
        Zunächst wird gezeigt, dass der Satz nur für $a_n = 1$ aus~\ref{subsec:algebraic-numbers} zu zeigen ist.}
    
    \subsection{Normieren des Polynoms}
        \label{subsec:norm-poly}
        \textrm{Angenommen, der \hyperref[eq:svr]{\emph{Satz von Roth}} gelte.\ Dann folgt aus $f(\xi) = 0$ aus~\ref
        {subsec:algebraic-numbers} durch Multiplikation mit $a_n^{n-1}$ für $a_n \xi \coloneq \Xi$:}
        \begin{equation*}
            0 = \Xi^n + a_{n-1} \Xi^{n-1} + \dsum{a_{n-2}a_n \Xi^{n-2}}{a_n^{n-1} a_0}
        \end{equation*}
        \textrm{und nach Multiplikation mit $\left| a_n \right|$ und geeigneter Abschätzung von~\eqref{eq:svr} gilt:}
        \begin{equation*}
            \left| \Xi - a_n \frac{p}{q} \right| < \left| a_n \right| q^{-(2+\delta)} < q^{-(2+\den{2}\delta)}
        \end{equation*}
        \textrm{für hinreichend große $q$.\ Da $\delta$ beliebig gewählt wurde, gilt der Satz somit nun für $\Xi$
            genau dann, wenn er für $\xi$ gilt. Somit gilt insgesamt oBdA:}
        \begin{equation}
            f(x) = x^n + \dsum{a_{n-1} x^{n-1}}{a_0} \sptext{mit} f(\xi) = 0 \sptext{und} \enum{a_{n-1}}{a_0}
            \in \mathbb{Z}. \label{eq:prereq}
        \end{equation}
        \newline
        \textrm{Seien im Übrigen}
        \begin{equation}
            n = \deg(f) \sptext{und} a = \max\{ 1, \enum{|a_{n-1}|}{|a_0|} \}. \label{eq:def-n-a}
        \end{equation}
        \newline
        Diese werden im weiteren Verlauf der Arbeit und des Beweises verwendet.
    
    
\subsection{Polynome}
    \label{subsec:polynomials}
    \textrm{In diesem Abschnitt werden Polynome und weitere Begriffe definiert, die in diesem Kapitel verwendet werden.
    Darauf folgen zwei Lemmata, die diese jeweils kurz näher beleuchten.}
    
    \subsubsection{Definition der Polynome}
        \label{subsubsec:def-poly}
        Es werden Polynome der folgenden Form behandelt:
        \begin{equation}
            R : \mathbb{R}^m \rightarrow \mathbb{R},\
            R(\enum{x_1}{x_m}) = \sum_{\substack{0 \leq j_\mu \leq r_\mu \\ 1 \leq \mu \leq m}}
            c_{\enum{j_1}{j_m}} \cdot \spdots{x_1^{j_1}}{x_m^{j_m}} \label{eq:def-poly}
        \end{equation}
        Wobei $m \in \mathbb{N}$, $r_\mu$ der Grad des Polynoms in $x_\mu$ und $c_{j_1, \dots, j_m} \in
        \mathbb{R}$ und seien.
        \newline \newline
        \emph{Beispiel}: Seien $m = 2,\ r_1 = 2 \sptext{und} r_2 = 1$.
        \newline
        \textrm{Das zugehörige Polynom nach obiger Definition sieht folgendermaßen aus:}
        \begin{equation*}
            R_{Bsp}(x_1, x_2) = c_{0,0} x_1^0 x_2^0 + c_{0,1} x_1^0 x_2^1 + c_{1,0} x_1^1 x_2^0 + c_{1,1} x_1^1
            x_2^1 + c_{2, 0} x_1^2 x_2^0 + c_{2,1} x_1^2 x_2^1
        \end{equation*}
        Nach konkreter Definition der Koeffizienten ergibt sich beispielsweise:
        \begin{equation*}
            R_{Bsp}(x_1, x_2) = 7 - \sqrt{\pi} x_2^1 + \frac{e}{\sqrt[3]{\gamma}} x_1^2 x_2^1
        \end{equation*}
    
    \subsubsection{Der Inhalt eines Polynoms}
        \label{subsubsec:content-poly}
        Sei R ein Polynom nach \hyperref[subsubsec:def-poly]{obiger Definition}.\ Dann sei der \emph{Inhalt} eines
        Polynoms wie folgt definiert:
        \begin{equation}
            \content{R} \coloneq \max\{|c_{\enum{j_1}{j_m}}| \} \  \forall \  0 \leq j_\mu \leq r_\mu, \
            1 \leq \mu \leq m \label{eq:def-content}
        \end{equation}
    
    \subsubsection{Restpolynom}
        \label{subsubsec:def-rempoly}
        Sei R ein Polynom nach \hyperref[subsubsec:def-poly]{obiger Definition} und seien $\enum{i_1}{i_m} \in
        \mathbb{N}_0$.\ Es wird
        \begin{equation}
            R_{\enum{i_1}{i_m}} = \den{\enum{i_1!}{i_m!}} \frac{\partial^{\dsum{i_1}{i_m}}}{
                \spdots{\partial x_1^{i_1}}{\partial x_m^{i_m}}} (R) \label{eq:def-rempoly}
        \end{equation}
        das \emph{Restpolynom} von R genannt.
    
    \subsubsection{Index eines Polynoms}
        \label{subsubsec:def-index}
        \textrm{Sei R ein Polynom nach \hyperref[subsubsec:def-poly]{obiger Definition}, $\alpha \in \mathbb{R}^m$
            und $s \in \mathbb{N}^m$.\ $I$ heißt \emph{Index} von R an der Stelle $\alpha$ bezüglich $s$, genau dann,
            wenn gilt:}
        \begin{equation}
            I \coloneq \ind(R) \coloneq \min_{(\enum{i_1}{i_m}) \in \mathbb{N}_0^m} \sum_{1 \leq \mu \leq m} \frac{i_\mu}
            {s_\mu}, \sptext{sodass} R_{i_1, \dots, i_m}(\alpha) \neq 0. \label{eq:def-index}
        \end{equation}
        Falls R verschwindet, wird konventionell $\ind(R) \coloneq \infty$ gesetzt.
        \newline
        \textrm{Dies erinnert an die \emph{Nullstellenordnung} einer Funktion (beispielsweise Nullstellen
        \emph{zweiter Ordnung} beziehungsweise \emph{zweifache} Nullstellen), lässt sich jedoch mit Hilfe des
        Vektors $s$ noch zusätzlich gewichten.
        \newline
        Zu diesem Begriff folgt wie erwähnt ein \hyperref[subsec:lemma2]{Lemma}, das weitere Eigenschaften
        darlegt.\ In dieser Arbeit wird diese Definition darüber hinaus jedoch nicht weiter verwendet; er findet lediglich
        in den Beweisen der weiteren Theorems im Beweis des \emph{Satzes von Roth} weitere Anwendung.}
    

    
    \subsection{Lemma 1}
        \label{subsec:lemma1}
        Sei R ein Polynom nach \hyperref[subsubsec:def-poly]{obiger Definition}.\ Dann gilt:
        \begin{enumerate}
            \item \textrm{$R$ hat ausschließlich Koeffizienten in $\mathbb{Z} \Rightarrow R_{\enum{i_1}{i_m}}$ hat
            auch ausschließlich Koeffizienten in $\mathbb{Z}$}
            \item \textrm{$R$ hat Grad $r_\mu$ in $x_\mu \Rightarrow R_{\enum{i_1}{i_m}}$ hat höchstens Grad $r_\mu
            - i_\mu$ in $x_\mu$ und verschwindet für $r_\mu < i_\mu$ ($\forall \  1 \leq \mu \leq m$)}
            \item $\content{R_{\enum{i_1}{i_m}}} \leq 2^{\dsum{r_1}{r_m}} \content{R}$
        \end{enumerate}
        \textrm{Diese Aussagen sind recht trivial, daher wird hier auf einen ausführlichen Beweis
        verzichtet und dem*der Leser*in überlassen.}
    
    \subsection{Lemma 2}
        \label{subsec:lemma2}
        Seien $\alpha \in \mathbb{R}^m,\ s \in \mathbb{N}^m$ und $R$ und $T$ Polynome nach \hyperref
        [subsubsec:def-poly]{obiger Definition}.\ Dann gilt:
        \begin{enumerate}
            \item $\ind(R_{\enum{i_1}{i_m}}) \geq \ind(R) - \sum \frac{i_\mu}{s_\mu}$
            \item $\ind(R_1 + R_2) \geq \min \{ \ind(R_1), \ind(R_2) \}$
            \item $\ind(R \cdot T) = \ind(R) +\ind(T)$
        \end{enumerate}
        \textrm{Da der Begriff des \emph{Index} in dieser Arbeit keine weitere Anwendung findet und die Aussagen
        außerdem nach kurzem Durchdenken ebenfalls recht trivial sind, wird auch hier auf einen ausführlichen Beweis
        verzichtet.}
    

    
    
\subsection{Polynome}
    \label{subsec:polynomials}
    \textrm{In diesem Abschnitt werden Polynome und weitere Begriffe definiert, die in diesem Kapitel verwendet werden.
    Darauf folgen zwei Lemmata, die diese jeweils kurz näher beleuchten.}
    
    \subsubsection{Definition der Polynome}
        \label{subsubsec:def-poly}
        Es werden Polynome der folgenden Form behandelt:
        \begin{equation}
            R : \mathbb{R}^m \rightarrow \mathbb{R},\
            R(\enum{x_1}{x_m}) = \sum_{\substack{0 \leq j_\mu \leq r_\mu \\ 1 \leq \mu \leq m}}
            c_{\enum{j_1}{j_m}} \cdot \spdots{x_1^{j_1}}{x_m^{j_m}} \label{eq:def-poly}
        \end{equation}
        Wobei $m \in \mathbb{N}$, $r_\mu$ der Grad des Polynoms in $x_\mu$ und $c_{j_1, \dots, j_m} \in
        \mathbb{R}$ und seien.
        \newline \newline
        \emph{Beispiel}: Seien $m = 2,\ r_1 = 2 \sptext{und} r_2 = 1$.
        \newline
        \textrm{Das zugehörige Polynom nach obiger Definition sieht folgendermaßen aus:}
        \begin{equation*}
            R_{Bsp}(x_1, x_2) = c_{0,0} x_1^0 x_2^0 + c_{0,1} x_1^0 x_2^1 + c_{1,0} x_1^1 x_2^0 + c_{1,1} x_1^1
            x_2^1 + c_{2, 0} x_1^2 x_2^0 + c_{2,1} x_1^2 x_2^1
        \end{equation*}
        Nach konkreter Definition der Koeffizienten ergibt sich beispielsweise:
        \begin{equation*}
            R_{Bsp}(x_1, x_2) = 7 - \sqrt{\pi} x_2^1 + \frac{e}{\sqrt[3]{\gamma}} x_1^2 x_2^1
        \end{equation*}
    
    \subsubsection{Der Inhalt eines Polynoms}
        \label{subsubsec:content-poly}
        Sei R ein Polynom nach \hyperref[subsubsec:def-poly]{obiger Definition}.\ Dann sei der \emph{Inhalt} eines
        Polynoms wie folgt definiert:
        \begin{equation}
            \content{R} \coloneq \max\{|c_{\enum{j_1}{j_m}}| \} \  \forall \  0 \leq j_\mu \leq r_\mu, \
            1 \leq \mu \leq m \label{eq:def-content}
        \end{equation}
    
    \subsubsection{Restpolynom}
        \label{subsubsec:def-rempoly}
        Sei R ein Polynom nach \hyperref[subsubsec:def-poly]{obiger Definition} und seien $\enum{i_1}{i_m} \in
        \mathbb{N}_0$.\ Es wird
        \begin{equation}
            R_{\enum{i_1}{i_m}} = \den{\enum{i_1!}{i_m!}} \frac{\partial^{\dsum{i_1}{i_m}}}{
                \spdots{\partial x_1^{i_1}}{\partial x_m^{i_m}}} (R) \label{eq:def-rempoly}
        \end{equation}
        das \emph{Restpolynom} von R genannt.
    
    \subsubsection{Index eines Polynoms}
        \label{subsubsec:def-index}
        \textrm{Sei R ein Polynom nach \hyperref[subsubsec:def-poly]{obiger Definition}, $\alpha \in \mathbb{R}^m$
            und $s \in \mathbb{N}^m$.\ $I$ heißt \emph{Index} von R an der Stelle $\alpha$ bezüglich $s$, genau dann,
            wenn gilt:}
        \begin{equation}
            I \coloneq \ind(R) \coloneq \min_{(\enum{i_1}{i_m}) \in \mathbb{N}_0^m} \sum_{1 \leq \mu \leq m} \frac{i_\mu}
            {s_\mu}, \sptext{sodass} R_{i_1, \dots, i_m}(\alpha) \neq 0. \label{eq:def-index}
        \end{equation}
        Falls R verschwindet, wird konventionell $\ind(R) \coloneq \infty$ gesetzt.
        \newline
        \textrm{Dies erinnert an die \emph{Nullstellenordnung} einer Funktion (beispielsweise Nullstellen
        \emph{zweiter Ordnung} beziehungsweise \emph{zweifache} Nullstellen), lässt sich jedoch mit Hilfe des
        Vektors $s$ noch zusätzlich gewichten.
        \newline
        Zu diesem Begriff folgt wie erwähnt ein \hyperref[subsec:lemma2]{Lemma}, das weitere Eigenschaften
        darlegt.\ In dieser Arbeit wird diese Definition darüber hinaus jedoch nicht weiter verwendet; er findet lediglich
        in den Beweisen der weiteren Theorems im Beweis des \emph{Satzes von Roth} weitere Anwendung.}
    

    
    \subsection{Lemma 1}
        \label{subsec:lemma1}
        Sei R ein Polynom nach \hyperref[subsubsec:def-poly]{obiger Definition}.\ Dann gilt:
        \begin{enumerate}
            \item \textrm{$R$ hat ausschließlich Koeffizienten in $\mathbb{Z} \Rightarrow R_{\enum{i_1}{i_m}}$ hat
                auch ausschließlich Koeffizienten in $\mathbb{Z}$}.
            \item \textrm{$R$ hat Grad $r_\mu$ in $x_\mu \Rightarrow R_{\enum{i_1}{i_m}}$ hat höchstens Grad $r_\mu
                - i_\mu$ in $x_\mu$ und verschwindet für $r_\mu < i_\mu$ ($\forall \  1 \leq \mu \leq m$)}.
            \item Es gilt $\content{R_{\enum{i_1}{i_m}}} \leq 2^{\dsum{r_1}{r_m}} \content{R}$\ .
        \end{enumerate}
        \textrm{Diese Aussagen sind recht trivial; der Beweis wird nur kurz vollzogen, da die Aussagen der Schritte im
        späteren Verlauf benötigt werden.}
        \newline
        \begin{proof}
            Die Aussagen folgen aus
            \begin{equation}
                R_{\enum{i_1}{i_m}}(\enum{x_1}{x_m}) = \sum_{\substack{0 \leq j_\mu \leq r_\mu \\ 1 \leq \mu \leq m}}
                \spdots{\binom{j_1}{i_1}}{\binom{j_m}{i_m}}c_{\enum{j_1}{j_m}} \cdot \spdots{x_1^{j_1}}{x_m^{j_m}},
                \label{eq:2.4}
            \end{equation}
            \textrm{wobei die Binominalkoeffizienten $\binom{j}{i}$ nach oben beschränkt sind durch:}
            \begin{equation}
                \binom{j}{i} \leq \sum_{0 \leq i \leq j} \binom{j}{i} = (1 + 1)^j \leq 2^r. \label{eq:2.5}
            \end{equation}
        \end{proof}
    
    \subsection{Lemma 2}
        \label{subsec:lemma2}
        Seien $\alpha \in \mathbb{R}^m,\ s \in \mathbb{N}^m$ und $R$ und $T$ Polynome nach \hyperref
        [subsubsec:def-poly]{obiger Definition}.\ Dann gilt:
        \begin{enumerate}
            \item $\ind(R_{\enum{i_1}{i_m}}) \geq \ind(R) - \sum \frac{i_\mu}{s_\mu}$
            \item $\ind(R_1 + R_2) \geq \min \{ \ind(R_1), \ind(R_2) \}$
            \item $\ind(R \cdot T) = \ind(R) +\ind(T)$
        \end{enumerate}
        \textrm{Da der Begriff des \emph{Index} in dieser Arbeit keine größere weitere Anwendung findet und die Aussagen
        außerdem nach kurzem Durchdenken ebenfalls recht trivial sind, wird auch hier auf einen ausführlichen Beweis
        verzichtet.}

    


    
\section{Konstruktion des Polynoms R (\emph{Theorem II})}
    \label{sec:th2}
    \textrm{In diesem Kapitel wird das Polynom $R$ konstruiert, das später im Beweis des \emph{Satzes von Roth}
    verwendet werden wird.\ Zum Beweis des \emph{Theorem II} werden drei Lemmata benötigt, von denen das \hyperref
    [subsec:lemma3]{Lemma 3} das zentrale Lemma ist, das am Ende die Bestimmung der Koeffizienten des Polynoms
    ermöglicht.}
    
    \subsection{Aussage des \emph{Theorem II}}
        \label{subsec:th2}
        Sei $\varepsilon > 0$ beliebig, $\xi$ algebraisch irrational, $n$ der Grad von $\xi$, sei
        \begin{equation}
            m \in \mathbb{Z} \text{, sodass} \  m > 8 n^2 \varepsilon^{-2} \label{eq:def-m}
        \end{equation}
        \textrm{und seien $\enum{r_1}{r_m} \in \mathbb{N}$.
        Dann existiert ein $R$ nach Definition in~\ref{subsubsec:def-poly} mit ganzzahligen Koeffizienten und höchstens
        Grad $r_\mu$ in $x_\mu \ (\forall \ 1 \leq \mu \leq m)$, für das gilt:}
        \begin{enumerate}
            \item $R$ verschwindet nicht
            \item \textrm{$\ind(R) \geq \den{2} m (1 - \varepsilon)$ an der Stelle $(\enum{\xi}{\xi})$ bezüglich
                $(\enum{r_1}{r_m})$}
            \item $\content{R} \leq \gamma^{\dsum{r_1}{r_m}}$ mit $\gamma = 4 (a + 1)$
        \end{enumerate}
        \textrm{mit a aus~\eqref{eq:def-a}.\ Hier ist vor allem wichtig, dass diese Aussagen für ein $m >$ eine Konstante
        abhängig von $\xi$ und $\varepsilon$ gelten.\ Der genaue Wert von $\gamma$ ist ebenfalls recht irrelevant, er
        kann von $\varepsilon$, $\xi$ oder $m$ abhängen.}
        \newline \newline
        \textrm{Die folgenden drei Lemmata werden für den \hyperref[subsec:proof-th2]{Beweis} benötigt:}
    
    
\subsection{Lemma 3}
    \label{subsec:lemma3}
    Seien $N, M \in \mathbb{N} \sptext{mit} N > M$ und seien
    \begin{equation*}
        L_j(\enum{z_1}{z_N}) = \sum_{1 \leq k \leq N} a_{jk} z_k \sptext{mit} 1 \leq j \leq M
    \end{equation*}
    $M$ viele $N$-Linearformen mit Koeffizienten $a_k \in \mathbb{Z}$ in $N$ vielen Variablen $z_k$.
    \textrm{Sei außerdem} $A \in \mathbb{N}$, sodass
    \begin{equation*}
        \left| a_{jk} \right| \leq A \ \ \forall \ 1 \leq j \leq M, 1 \leq k \leq N
    \end{equation*}
    \textrm{Dann besitzt das System $L = (\enum{L_1}{L_M})$ in den Variablen $\enum{z_1}{z_N}$ Lösungen in $\mathbb{Z}$, die nicht alle
    verschwinden und sodass gilt:}
    \begin{equation*}
        L_j = 0 \sptext{mit} 1 \leq j \leq M \sptext{und} \left| z_i \right| \leq Z = \left[ (NA)^{\frac{M}{N-M}} \right]
        \sptext{mit} 1 \leq k \leq n.
    \end{equation*}
    \begin{proof}
        Seien $N, M, A, Z$ wie oben beschrieben.\ Es gilt:
        \begin{equation*}
            NA < (Z + 1)^{\frac{N-M}{M}}
        \end{equation*}
        woraus folgt:
        \begin{equation*}
            NAZ+1 \leq NA(Z + 1) < (Z + 1)^{\frac{N}{M}}
        \end{equation*}
        \textrm{Für alle} $N$-Tupel $\zeta = (\enum{z_1}{z_N})$ mit
        \begin{equation}
            0 \leq z_k \leq Z \sptext{mit} 1 \leq k \leq N \label{eq:z_k-condition}
        \end{equation}
        gilt:
        \begin{equation*}
            \shine
        \end{equation*}
        \textrm{Somit ergeben sich für $L_j(z) \in \mathbb{Z}$ höchstens $NAZ+1$ viele verschiedene Werte und es existieren
        höchstens $(Z + 1)^N$ viele verschiedene $\zeta$, woraus sich jedoch höchstens $(Z + 1)^N > (NAZ+1)^M$ viele
        verschiedene $L$ ergeben.}
        \newline
        Somit existieren zwei verschiedene $M$-Tupel $\zeta_1, \zeta_2$, sodass gilt:
        \begin{equation*}
            L_j(\zeta_1) = L_j(\zeta_2) \sptext{mit} 1 \leq j \leq M.
        \end{equation*}
        mit $\zeta = \zeta_1 - \zeta_2$ folgt die Behauptung.
    \end{proof}

\subsection{Lemma 4}
    \label{subsec:lemma4}
    \textrm{Sei $\xi$ algebraisch irrational und $n$ dessen Grad. Für alle $l \in \mathbb{N}_0$ existieren
    Koeffizienten $a_{j,l} \in \mathbb{Z}$ mit $1 \leq j < n$, sodass gilt:}
    \begin{equation*}
        \xi^l = \dsum{a_{n-1, l}\xi^{n-1}}{a_{0,l}}
    \end{equation*}
    \textrm{und sind mit $a$ aus~\eqref{eq:def-a} folgendermaßen nach oben beschränkt:}
    \begin{equation*}
        \left| a_{a,l} \right| \leq (a + 1)^l
    \end{equation*}
    \begin{proof}
        \textrm{Für $l < n$ ist die Aussage offensichtlich. Für $l \geq n$ wird eine kurze Induktion aufgezogen:}
        \newline
        Induktionsanfang: $l = n$
        \begin{indentpar}
            Folgt durch Umformung aus $f(x) = 0$ aus~\ref{subsec:algebraic-numbers}.
        \end{indentpar}
        Induktionsvoraussetzung:
        \begin{indentpar}
            \textrm{Angenommen, die Behauptung gilt für ein $l > n$.}
        \end{indentpar}
        Induktionsschritt: $l \Rightarrow l+1$
        \begin{indentpar}
            Folgt mit Hilfe von
            \begin{equation*}
                \xi^{l+1} = \xi \cdot \xi^l = \dsum{a_{n-1, l}\xi^{n}}{a_{0,l} \xi}.
            \end{equation*}
        \end{indentpar}
        \textrm{Per Induktionsaxiom folgt die Aussage für alle $l \in \mathbb{N}$.}
    \end{proof}

\subsection{Lemma 5}
    \label{subsec:lemma5}
    \textrm{Seien $\enum{r_1}{r_m} \in \mathbb{N}$ und $0 < \lambda \in \mathbb{R}$.\ Dann gibt es höchstens}
    \begin{equation*}
        \frac{\sqrt{2m}}{\lambda} \spdots{(r_1 + 1)}{(r_m +1)}
    \end{equation*}
    \textrm{viele $m$-Tupel $(\enum{i_1}{i_m})$, die die folgende Ungleichung erfüllen:}
    \begin{equation*}
        \sum_{\substack{0 \leq i_\mu \leq r_\mu \\ 1 \leq \mu \leq m}} \frac{i_\mu}{r_\mu} \leq \den{2} (m - \lambda)
    \end{equation*}
    \begin{proof}
        bla. \shine
    \end{proof}
    

    
    \subsection{Beweis des \emph{Theorem II}}
        \label{subsec:proof-th2}
        \textrm{Wie in der \hyperref[sec:th2]{Einleitung des Kapitels} erwähnt, läuft der Beweis darauf hinaus, das
        \hyperref[subsec:lemma3]{Lemma 3} anzuwenden.\ Die nötige Vorarbeit wird in erster Linie mit den Lemmata
        \hyperref[subsec:lemma4]{4} und \hyperref[subsec:lemma5]{5} geleistet.}
        \begin{namedproof}{\emph{\hyperref[subsec:th2]{Theorem II}}}
            Sei $R$ ein Polynom nach Definition in~\ref{subsubsec:def-poly} und seien $\varepsilon,\ \xi,\ n,\ m \sptext
            {und} \enum{r_1}{r_m}$ wie in~\ref{subsec:th2} beschrieben.
            \newline
            Zu zeigen ist, dass die $\spdots{(r_1+1)}{(r_m+1)} = N$ vielen Koeffizienten von R existieren.
            \newline 
            Dazu soll gelten $\forall \ \enum{i_1}{i_m} \in \mathbb{N}$:
            \begin{equation}
                R_{\enum{i_1}{i_m}}(\enum{\xi}{\xi}) = 0 \label{eq:rem-poly-condition}
            \end{equation}
            sodass ebenfalls gilt:
            \begin{equation*}
                \sum_{1 \leq \mu \leq m} \frac{i_\mu}{r_\mu} \leq \den{2} m(1 - \varepsilon)
            \end{equation*}
            Da das Polynom verschwindet und~\eqref{eq:rem-poly-condition} somit trivial ist, falls $i_\mu > r_\mu$, sei
            also $i_\mu \leq r_\mu$ ($\forall \ 1 \leq \mu \leq m$).
            
        \end{namedproof}
    

        
\end{document}