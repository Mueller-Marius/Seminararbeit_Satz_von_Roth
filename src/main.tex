%! Author = gurki
%! Date = 08.08.22

% Preamble
\documentclass[11pt]{article}

\title{Seminararbeit zum Vortrag ``Satz von Roth I'' im Seminar ``Analysis'' bei Prof. Dr. Hein}
\author{Marius Müller}
\date{Juli 2022}
% Packages
\usepackage{amsmath}
\usepackage{amsfonts}
\usepackage{mathtools}
\usepackage{hyperref}
\usepackage{stmaryrd}

\newcommand{\overbar}[1]{\mkern -3mu\overline{\mkern+3mu#1\mkern+4mu}\mkern -3mu}
\newcommand{\content}[1]{\left| \overbar{#1} \right|}

% Document
\begin{document}

    \maketitle

    \vfill

    \begin{abstract}
        \textrm{Diese Arbeit behandelt den Satz von Thue-Siegel-Roth,
        der mittels des Irrationalitätsmaßes eine Aussage über die Irrationalität algebraisch irrationaler Zahlen liefert.
        \newline
        In dieser Arbeit wird zur Thematik hingeführt, die nötigen Grundlagen behandelt und das erste Theorem im Beweis
        des Satzes von Roth erklärt und bewiesen.}
    \end{abstract}

    \newpage

    \tableofcontents

    \newpage

    \section{Einleitung}
    \label{sec:intro}
        \textrm{Der \emph{Satz von Thue-Siegel-Roth} (im Folgenden kurz \emph{Satz von Roth} genannt) wurde erstmals von
        Klaus Friedrich Roth bewiesen, der im Jahre 1958 für diesen Meilenstein die \emph{Fields-Medallie} verliehen bekam.
        \newline
        Diese Arbeit ist eng an das Kapitel VI des Buches ``An Introduction To Diophantine Approximation'' von
        John William Scott Cassels aus dem Jahre 1957 angelehnt.
        \newline
        Der Beweis des \emph{Satzes von Roth} gliedert sich hier in drei Theorems.\ Von diesen wird in dieser
        Arbeit der erste Satz, das \emph{Theorem II}, beschrieben, erklärt und bewiesen (der \emph{Satz von Roth} selbst
        ist hier das \emph{Theorem I}; der Übersichtlichkeit halber wird sich an die Nummerierung der Quelle gehalten.)}

    \section{Motivation des Themas}
    \label{sec:motivation}
        \textrm{Das sogenannte \emph{Irrationalitätsmaß} quantifiziert die Irrationalität einer reellen Zahl. Dazu wird
        die folgende Definition verwendet:}

        \subsection{\textrm{Das Irrationalitätsmaß}}
        \label{subsec:irr-measure}
            \textrm{Sei $x \in \mathbb{R}$ beliebig. Sei $M$ die Menge aller $\mu \in \mathbb{R}$, sodass die Ungleichung
            \begin{equation*}
                0 < |x - \frac{p}{q}| < \frac{1}{q^\mu}
            \end{equation*}
            nur endlich viele Lösungen in $p \in \mathbb{Z}, q \in \mathbb{N}$
            besitzt.\ Dann heißt
            \begin{equation*}
                \mu(x) \coloneqq \inf(M)
            \end{equation*}
            das \emph{Irrationalitätsmaß} von $x$.
            \newline
            Die folgenden Beispiele illustrieren diese Definition.}

        \subsection{\textrm{Beispiele zum Irrationalitätsmaß}}
        \label{subsec:examples-irr-measure}
            \begin{itemize}
                \item \textrm{Für $x \in \mathbb{Q}$ gilt: $\mu(x) = 1$}
                \item \textrm{Für irrationale $x$ wurde gezeigt, dass gilt: $\mu(x) \geq 2$}
                \item \textrm{Für die \emph{eulersche Zahl} $e$ gilt $\mu(e) = 2$}
                \item \textrm{Das Irrationalitätsmaß der Kreiszahl $\pi$ ist bisher unbekannt.\ Der neuste Fortschritt
                            setzt die obere Schranke bei $\mu(\pi) \leq 7,1032\dots$ fest.}
            \end{itemize}

        \subsection{Zentrale Fragestellung des \emph{Satzes von Roth}}
        \label{subsec:question}
            \textrm{Es stellt sich nach den oben genannten Beispielen die Frage, ob auch alle irrationale Zahlen dasselbe
            Irrationalitätsmaß besitzen. Hier liefert der \emph{Satz von Roth} eine teilweise Antwort:
            \newline
            Das Irrationalitätsmaß aller \emph{algebraischen} irrationalen Zahlen ist genau zwei.}

    \section{Grundlagen und Voraussetzungen}
    \label{sec:basics}
        Algebraische Zahlen, formale Aussage des Satzes (\emph{Theorem I}), oBdA $a_n = 1$ und Sachen zu Polynomen
        
        \subsection{algebraische Zahlen}
        \label{subsec:algebraic-numbers}
            \textrm{Sei $z \in \mathbb{C}$. $z$ heißt \emph{algebraisch} genau dann, wenn gilt:}
            \begin{equation}
                \exists f \in \mathbb{Q}[x] : f(z) = 0 \label{eq:def-algebraic}
            \end{equation}
            \textrm{d.h.\ falls z eine Lösung eines Polynoms mit rationalen Koeffizienten ist.\ OBdA kann angenommen
            werden, dass $f \in \mathbb{Q}[x]$, da sich die Gleichung $f(x) = 0$ mit dem Produkt der Nenner der
            Koeffizienten der Form $a_n = \frac{p_n}{q_n}$, d.h.\ mit $\prod_{i=1}^k q_i$ multiplizieren lässt, wodurch
            alle Koeffizienten ganzzahlig werden, die Gleichung und damit auch das resultierende Polynom jedoch dieselben
            Lösungen bzw.\ Nullstellen besitzen.
            \newline
            Weiterhin lässt sich oBdA annehmen, dass für den Koeffizienten der höchsten Potenz von $x$ gilt: $a_n \neq 0$.}
        
        \subsection{Satz von Roth (\emph{Theorem I})}
        \label{subsec:th1}
            Sei $\xi \in \mathbb{R}$ algebraisch irrational und $\delta > 0$ beliebig.\ Dann besitzt die Ungleichung
            \begin{equation}
                0 < | \xi - \frac{p}{q} | < q^{-(2+\delta)} \label{eq:svr}
            \end{equation}
            \textrm{nur endlich viele Lösungen in $p \in \mathbb{Z}$, $q \in \mathbb{N}$.
            \newline \newline
            Zunächst wird gezeigt, dass der Satz nur für $a_n = 1$ aus  zu zeigen ist.}
        
        \subsection{Normieren des Polynoms}
        \label{subsec:norm-poly}
            \textrm{Angenommen, der \hyperref[eq:svr]{\emph{Satz von Roth}} gelte.\ Dann gilt für $a_n \xi \coloneq \Xi$:}
            \begin{equation*}
                0 = \Xi^n + a_{n-1} \Xi^{n-1} + \dots + a_n^{n-1} a_0
            \end{equation*}
            \textrm{und nach Multiplikation mit $\left| a_n \right|$ und geeigneter Abschätzung von~\eqref{eq:svr} gilt:}
            \begin{equation*}
                \left| \Xi - a_n \frac{p}{q} \right| < \left| a_n \right| q^{-(2+\delta)} < q^{-(2+\frac{1}{2}\delta)}
            \end{equation*}
            \textrm{für hinreichend große $q$.\ Da $\delta$ beliebig gewählt wurde, gilt der Satz somit nun für $\Xi$
            genau dann, wenn er für $\xi$ gilt. Somit gilt insgesamt oBdA:}
            \begin{equation}
                f(x) = x^n + a_{n-1} x^{n-1} + \dots + a_0 \  \text{mit} \  f(\xi) = 0 \  \text{und} \  a_{n-1}, \dots,
                a_0 \in \mathbb{Z}. \label{eq:prereq}
            \end{equation}
            \newline
            \textrm{Seien im Übrigen}
            \begin{equation}
                n = \deg(f) \  \text{und} \  a = \max\{ 1, | a_{n-1} |, \dots, | a_0 | \}. \label{eq:def-a}
            \end{equation}
            \newline
            Diese werden im weiteren Verlauf der Arbeit und des Beweises verwendet werden.

        \subsection{Polynome}
        \label{subsec:polynomials}
            \textrm{In diesem Abschnitt werden Polynome und weitere Begriffe definiert, die in diesem Kapitel verwendet werden.
            Darauf folgen zwei Lemmata, die diese jeweils kurz näher beleuchten.}
        
            \subsubsection{Definition der Polynome}
            \label{subsubsec:def-poly}
                Es werden Polynome der folgenden Form behandelt:
                \begin{equation}
                    R : \mathbb{R}^m \rightarrow \mathbb{R},\
                    R(x_1, \dots, x_m) = \sum_{\substack{0 \leq j_\mu \leq r_\mu \\ 1 \leq \mu \leq m}}
                    c_{j_1, \dots, j_m} \cdot x_1^{j_1} \dots x_m^{j_m}. \label{eq:def-poly}
                \end{equation}
                Wobei $m \in \mathbb{N}$, $r_\mu$ der Grad des Polynoms in $x_\mu$ und $c_{j_1, \dots, j_m} \in
                \mathbb{R}$ und seien.
                \newline \newline
                \emph{Beispiel}: Seien $m = 2,\ r_1 = 2 \  \text{und} \  r_2 = 1$.
                \newline
                \textrm{Das zugehörige Polynom nach obiger Definition sieht folgendermaßen aus:}
                \begin{equation*}
                    R_{Bsp}(x_1, x_2) = c_{0,0} x_1^0 x_2^0 + c_{0,1} x_1^0 x_2^1 + c_{1,0} x_1^1 x_2^0 + c_{1,1} x_1^1
                    x_2^1 + c_{2, 0} x_1^2 x_2^0 + c_{2,1} x_1^2 x_2^1
                \end{equation*}
                Nach konkreter Definition der Koeffizienten ergibt sich beispielsweise:
                \begin{equation*}
                    R_{Bsp}(x_1, x_2) = 7 - \sqrt{\pi} x_2^1 + \frac{e}{\sqrt[3]{\gamma}} x_1^2 x_2^1
                \end{equation*}
            
            \subsubsection{Der Inhalt eines Polynoms}
            \label{subsubsec:content-poly}
                Sei R ein Polynom nach \hyperref[subsubsec:def-poly]{obiger Definition}.\ Dann sei der \emph{Inhalt} eines
                Polynoms wie folgt definiert:
                \begin{equation}
                     \content{R} \coloneq \max\{|c_{j_1, \dots, j_m}| \} \  \forall \  0 \leq j_\mu \leq r_\mu, \
                     1 \leq \mu \leq m \label{eq:def-content}
                \end{equation}
            
            \subsubsection{Restpolynom}
            \label{subsubsec:def-rempoly}
                Sei R ein Polynom nach \hyperref[subsubsec:def-poly]{obiger Definition}.\ Es wird
                \begin{equation}
                    R_{i_1, \dots, i_m} = \frac{1}{i_1! \dots i_m!} \frac{\partial^{i_1 + \dots + i_m}}{
                    \partial x_1^{i_1} \dots \partial x_m^{i_m}} (R) \label{eq:def-rempoly}
                \end{equation}
                das \emph{Restpolynom} von R genannt.
            
            \subsubsection{Index eines Polynoms}
            \label{subsubsec:def-index}
                \textrm{Sei R ein Polynom nach \hyperref[subsubsec:def-poly]{obiger Definition}, $\alpha \in \mathbb{R}^m$
                und $s \in \mathbb{N}^m$.\ $I$ heißt \emph{Index} von R an der Stelle $\alpha$ bezüglich $s$, genau dann,
                wenn gilt:}
                \begin{equation}
                    I \coloneq ind(R) = \min \sum_{1 \leq \mu \leq m} \frac{i_\mu}{s_\mu} \text{, sodass} \ R_{i_1, \dots, i_m}(\alpha)
                    \neq 0. \label{eq:def-index}
                \end{equation}
                \textrm{Dies erinnert an die \emph{Nullstellenordnung} einer Funktion, lässt sich jedoch mit Hilfe von $s$
                noch zusätzlich gewichten.}
                \newline
                \textrm{Zu diesem Begriff folgt wie erwähnt ein \hyperref[subsec:lemma2]{Lemma}, das weitere Eigenschaften darlegt.\ In
                dieser Arbeit wird diese Definition jedoch nicht weiter verwendet; er findet lediglich in den Beweisen
                der weiteren Theorems im Beweis des \emph{Satzes von Roth} weitere Anwendung.}
        
        \subsection{Lemma 1}
        \label{subsec:lemma1}
            Lemma 1 mit Beweis (bisschen was zum Taylor Ableitungen Bums)
        
        \subsection{Lemma 2}
        \label{subsec:lemma2}
            Lemma 2 maybe mit Beweis? (paar Eigenschaften zum Index)

    \section{Konstruktion des Polynoms R (\emph{Theorem II})}
    \label{sec:th2}
        Zwischengelaber mit Kontext, Struktur, Sinn im Beweis, etc.

        \subsection{Aussage des \emph{Theorem II}}
        \label{subsec:th2}
            \textrm{Satz 2 erklären}

        \subsection{Lemma 3}
        \label{subsec:lemma3}
            Lemma 3 halt, maybe mit Beweis?

        \subsection{Lemma 4}
        \label{subsec:lemma4}
            Lemma 4 halt, maybe mit Beweis?

        \subsection{Lemma 5}
        \label{subsec:lemma5}
            Lemma 5 halt, maybe mit Beweis?

        \subsection{Beweis des \emph{Theorem II}}
        \label{subsec:proof-th-2}
            letztendlicher Beweis des Theorem II
        
\end{document}