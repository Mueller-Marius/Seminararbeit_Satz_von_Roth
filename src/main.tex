%! Author = gurki
%! Date = 08.08.22

% Preamble
\documentclass[11pt]{article}

\title{Seminararbeit zum Vortrag ``Satz von Roth I'' im Seminar ``Analysis'' bei Prof. Dr. Hein}
\author{Marius Müller}
\date{Juli 2022}


% Packages
\usepackage{amsmath}
\usepackage{amsfonts}
\usepackage{mathtools}
\usepackage{hyperref}
\usepackage{amssymb}
\usepackage[tikz]{mdframed}
\usepackage{wasysym}
\usepackage{color}

\setlength{\parindent}{0pt}

\DeclareMathOperator{\ind}{ind} %index

\newcommand{\content}[1]{\begin{tabular}{|c|}\hline \!#1\! \end{tabular}} %for Polynomials
\newcommand{\enum}[2]{#1, \dots, #2}
\newcommand{\dsum}[2]{#1 + \dots + #2}
\newcommand{\spdots}[2]{#1 \dots #2}
\newcommand{\sptext}[1]{\ \textrm{#1}\ }
\newcommand{\den}[1]{\frac{1}{#1}}
\newcommand{\cbrt}[1]{\sqrt[3]{#1}}
\newcommand{\shine}{\textcolor{cyan}{\textbf{\varhexagon}}}
\newcommand{\spaceoperator}{\forall \ }

\newmdenv[
    linewidth=0pt,
    leftmargin=17pt,
    skipabove=0pt,
    skipbelow=-8pt
]{indentpar}

\newenvironment{proof}
    {\newline \emph{Beweis}. \newline}
    {\hfill $Q.e.d.$}
\newenvironment{namedproof}[1]
    {\newline \emph{Beweis}(#1): \newline}
    {\hfill $Q.e.d.$}
\newenvironment{shinyproof}
    {\newline \emph{Beweis.} \newline}
    {\hfill \shine}
    


% Document
\begin{document}

    \maketitle

    \vfill

    \begin{abstract}
        \noindent \textrm{Diese Arbeit behandelt den \emph{Satz von Thue-Siegel-Roth}, der mittels des Irrationalitätsmaßes
        eine Aussage über die Irrationalität algebraisch irrationaler Zahlen liefert.
        \newline
        In dieser Arbeit wird zur Thematik hingeführt, die nötigen Grundlagen behandelt und das erste Theorem im Beweis
        des Satzes erklärt und bewiesen.}
    \end{abstract}

    \newpage

    \tableofcontents

    \newpage

    \section{Einleitung}
    \label{sec:intro}
        \textrm{Der \emph{Satz von Thue-Siegel-Roth} (im Folgenden kurz \emph{Satz von Roth} genannt) wurde erstmals von
        \emph{Klaus Friedrich Roth} bewiesen, der im Jahre 1958 für diesen Meilenstein die \emph{Fields-Medallie}
        verliehen bekam.
        \newline
        Diese Arbeit ist eng an das Kapitel VI des Buches ``An Introduction To Diophantine Approximation'' von
        John W. S. Cassels aus 1957 angelehnt.
        \newline
        Der Beweis des \emph{Satzes von Roth} gliedert sich hier in drei Theorems.\ Von diesen wird in dieser
        Arbeit der erste Satz, das \emph{Theorem II}, beschrieben, erklärt und bewiesen (der \emph{Satz von Roth} selbst
        ist hier das \emph{Theorem I}; der Übersichtlichkeit halber wird sich an die Nummerierung der Quelle gehalten;
        die Notation wurde jedoch stellenweise abgeändert.)}

    
\section{Motivation des Themas}
    \label{sec:motivation}
    \textrm{Das sogenannte \emph{Irrationalitätsmaß} quantifiziert die Irrationalität einer reellen Zahl. Dazu wird
    die folgende Definition verwendet:}
    
    \subsection{\textrm{Das Irrationalitätsmaß}}
        \label{subsec:irr-measure}
        \textrm{Sei $x \in \mathbb{R}$ beliebig. Sei $M$ die Menge aller $\mu \in \mathbb{R}$, sodass die Ungleichung
            \begin{equation*}
                0 < \left| x - \frac{p}{q} \right| < \frac{1}{q^\mu}
            \end{equation*}
            nur endlich viele Lösungen in $p \in \mathbb{Z}, q \in \mathbb{N}$
            besitzt.\ Dann heißt
            \begin{equation*}
                \mu(x) \coloneqq \inf(M)
            \end{equation*}
            das \emph{Irrationalitätsmaß} von $x$.
            \newline
            Die folgenden Beispiele illustrieren diese Definition.}
    
    \subsection{\textrm{Beispiele zum Irrationalitätsmaß}}
        \label{subsec:examples-irr-measure}
        \begin{itemize}
            \item \textrm{Für $x \in \mathbb{Q}$ gilt: $\mu(x) = 1$}
            \item \textrm{Für irrationale $x$ wurde gezeigt, dass gilt: $\mu(x) \geq 2$}
            \item \textrm{Für die \emph{eulersche Zahl} $e$ gilt $\mu(e) = 2$}
            \item \textrm{Das Irrationalitätsmaß der Kreiszahl $\pi$ ist bisher unbekannt.\ Der neuste Fortschritt
            setzt die obere Schranke bei $\mu(\pi) \leq 7,1032\dots$ fest.}
        \end{itemize}
    
    \subsection{Zentrale Fragestellung des \emph{Satzes von Roth}}
        \label{subsec:question}
        \textrm{Es stellt sich nach den oben genannten Beispielen die Frage, ob auch alle irrationale Zahlen dasselbe
        Irrationalitätsmaß besitzen. Hier liefert der \emph{Satz von Roth} eine teilweise Antwort:
        \newline
        Das Irrationalitätsmaß aller \emph{algebraisch} irrationalen Zahlen ist genau zwei.}
    

    
    
\subsection{algebraische Zahlen}
    \label{subsec:algebraic-numbers}
    \textrm{Eine komplexe Zahl $z \in \mathbb{C}$ heißt \emph{algebraisch} genau dann, wenn gilt:}
    \begin{equation}
        \exists f \in \mathbb{Q}[x] : f(z) = 0 \label{eq:def-algebraic}
    \end{equation}
    \textrm{d.h.\ falls z eine Lösung eines Polynoms mit rationalen Koeffizienten $a_k~=~\frac{p_k}{q_k}$ mit $p_k \in
    \mathbb{Z},\ q_k \in \mathbb{N}\ (\forall \ 1 \leq k \leq n)$ ist.
    \newline
    Die Zahl $n = \deg f$ heißt \emph{Grad} der algebraischen Zahl $z$.
    \newline
    OBdA kann angenommen werden, dass $f \in \mathbb{Z}[x]$, da sich die Gleichung $f(x) = 0$ mit dem Produkt der Nenner
    der Koeffizienten multiplizieren lässt, wodurch alle Koeffizienten ganzzahlig werden, die Gleichung und damit auch
    das resultierende Polynom jedoch dieselben Lösungen bzw.\ Nullstellen besitzen.\ Weiterhin lässt sich oBdA annehmen,
    dass für den Koeffizienten der höchsten Potenz von $x$ gilt: $a_n \neq 0$.}

\subsection{Der Satz von Roth (\emph{Theorem I})}
    \label{subsec:th1}
    Sei $\xi \in \mathbb{R}$ algebraisch irrational und $\delta > 0$ beliebig.\ Dann besitzt die Ungleichung
    \begin{equation}
        0 < \left| \xi - \frac{p}{q} \right| < q^{-(2+\delta)} \label{eq:svr}
    \end{equation}
    \textrm{nur endlich viele Lösungen in $p \in \mathbb{Z}$, $q \in \mathbb{N}$.
    \newpage
    \textrm{Hiermit liefert der Satz die obere Schranke $2$ für das Irrationalitätsmaß algebraisch irrationaler Zahlen;
    zusammen mit der unteren Schranke von ebenfalls $2$ gilt somit $\mu(x) = 2$ für alle algebraisch irrationalen Zahlen
    $x$.}
    \newline \newline
    Zunächst wird gezeigt, dass der Satz nur für $a_n = 1$ aus~\ref{subsec:algebraic-numbers} zu zeigen ist.}

\subsection{Normieren des Polynoms einer algebraischen Zahl}
    \label{subsec:norm-poly}
    \textrm{Angenommen, der \hyperref[eq:svr]{\emph{Satz von Roth}} gelte.\ Dann folgt aus $f(\xi) = 0$ aus~\ref
    {subsec:algebraic-numbers} durch Multiplikation mit $a_n^{n-1}$ für $a_n \xi \coloneq \Xi$:}
    \begin{equation*}
        0 = \Xi^n + a_{n-1} \Xi^{n-1} + \dsum{a_{n-2}a_n \Xi^{n-2}}{a_n^{n-1} a_0}
    \end{equation*}
    \textrm{und nach Multiplikation mit $\left| a_n \right|$ und geeigneter Abschätzung von~\eqref{eq:svr} gilt:}
    \begin{equation*}
        \left| \Xi - a_n \frac{p}{q} \right| < \left| a_n \right| q^{-(2+\delta)} < q^{-(2+\den{2}\delta)}
    \end{equation*}
    \textrm{für hinreichend große $q$.\ Da $\delta$ beliebig gewählt wurde, gilt der Satz somit nun für $\Xi$ genau dann,
    wenn er für $\xi$ gilt. Somit gilt insgesamt oBdA:}
    \begin{equation}
        f(x) = x^n + \dsum{a_{n-1} x^{n-1}}{a_0} \sptext{mit} f(\xi) = 0 \sptext{und} \enum{a_{n-1}}{a_0}
        \in \mathbb{Z}. \label{eq:prereq}
    \end{equation}
    \newline
    \textrm{Sei im Übrigen}
    \begin{equation}
        a = \max\{ 1, \enum{|a_{n-1}|}{|a_0|} \}. \label{eq:def-a}
    \end{equation}
    \newline
    Dies wird im weiteren Verlauf der Arbeit und des Beweises verwendet.
    


    
\section{Konstruktion des Polynoms R (\emph{Theorem II})}
    \label{sec:th2}
    \textrm{In diesem Kapitel wird das Polynom $R$ konstruiert, das später im Beweis des \emph{Satzes von Roth}
    verwendet werden wird.\ Zum Beweis des \emph{Theorem II} werden drei Lemmata benötigt, von denen das \hyperref
    [subsec:lemma3]{Lemma~3} das zentrale Lemma ist, das am Ende die Bestimmung der Koeffizienten des Polynoms
    ermöglicht.}
    
    \subsection{Aussage des \emph{Theorem II}}
        \label{subsec:th2}
        Sei $\varepsilon > 0$ beliebig, $\xi$ algebraisch irrational, $n$ der Grad von $\xi$, sei
        \begin{equation}
            m \in \mathbb{Z} \text{, sodass} \  m > 8 n^2 \varepsilon^{-2} \label{eq:def-m}
        \end{equation}
        \textrm{und seien $\enum{r_1}{r_m} \in \mathbb{N}$.
        \newline
        Dann existiert ein Polynom $R$ nach Definition in~\ref{subsubsec:def-poly} mit ganzzahligen Koeffizienten und
        höchstens Grad $r_\mu$ in $x_\mu \ (\forall \ 1 \leq \mu \leq m)$, für das gilt:}
        \begin{enumerate}
            \item $R$ verschwindet nicht \label{st:R-st}
            \item \textrm{$\ind(R) \geq \den{2} m (1 - \varepsilon)$ an der Stelle $(\enum{\xi}{\xi})$ bezüglich
                $(\enum{r_1}{r_m})$} \label{st:ind-st}
            \item $\content{R} \leq \gamma^{\dsum{r_1}{r_m}}$ mit $\gamma = 4 (a + 1)$ \label{st:cont-st}
        \end{enumerate}
        \textrm{mit a aus~\eqref{eq:def-a}.\ Hier ist vor allem wichtig, dass diese Aussagen für ein $m >$ eine Konstante
        abhängig von $\xi$ und $\varepsilon$ gelten.\ Der genaue Wert von $\gamma$ ist ebenfalls recht irrelevant, er
        kann von $\varepsilon$, $\xi$ oder $m$ abhängen.}
        \newline \newline
        \textrm{Die folgenden drei Lemmata werden für den \hyperref[subsec:proof-th2]{Beweis} benötigt:}
    
    
\subsection{Lemma 3}
    \label{subsec:lemma3}
    Seien $N, M \in \mathbb{N} \sptext{mit} N > M$ und seien
    \begin{equation*}
        L_j(\enum{z_1}{z_N}) = \sum_{1 \leq k \leq N} a_{jk} z_k \sptext{mit} 1 \leq j \leq M
    \end{equation*}
    $M$ viele $N$-Linearformen mit Koeffizienten $a_k \in \mathbb{Z}$ in $N$ vielen Variablen $z_k$.
    \textrm{Sei außerdem} $A \in \mathbb{N}$, sodass
    \begin{equation*}
        \left| a_{jk} \right| \leq A \ \ \forall \ 1 \leq j \leq M, 1 \leq k \leq N
    \end{equation*}
    \textrm{Dann besitzt das System $L = (\enum{L_1}{L_M})$ in den Variablen $\enum{z_1}{z_N}$ Lösungen in $\mathbb{Z}$,
    die nicht alle verschwinden und sodass gilt:}
    \begin{equation*}
        L_j = 0 \sptext{mit} 1 \leq j \leq M \sptext{und} \left| z_i \right| \leq Z = \left[ (NA)^{\frac{M}{N-M}} \right]
        \sptext{mit} 1 \leq k \leq n.
    \end{equation*}
    \begin{proof}
        Seien $N, M, A, Z$ wie oben beschrieben.\ Sei $- B_j$ die Summe der negativen und $P_j$ die Summe der positiven
        Koeffizienten in $L_j(z)$.
        \newline
        Es gilt:
        \begin{equation*}
            NA < (Z + 1)^{\frac{N-M}{M}}
        \end{equation*}
        woraus folgt:
        \begin{equation*}
            NAZ+1 \leq NA(Z + 1) < (Z + 1)^{\frac{N}{M}}
        \end{equation*}
        \textrm{Für alle} $N$-Tupel $\zeta = (\enum{z_1}{z_N})$ mit
        \begin{equation}
            0 \leq z_k \leq Z \sptext{mit} 1 \leq k \leq N \label{eq:z_k-condition}
        \end{equation}
        gilt:
        \begin{equation*}
            - B_j Z \leq L_j(z) \leq P_j Z \sptext{und} B_j + P_j \leq N A
        \end{equation*}
        \textrm{Somit ergeben sich für $L_j(z) \in \mathbb{Z}$ höchstens $NAZ+1$ viele verschiedene Werte. Es existieren
        auch höchstens $(Z + 1)^N$ viele verschiedene $\zeta$, woraus sich jedoch höchstens $(Z + 1)^N > (NAZ+1)^M$ viele
        verschiedene Systeme $L$ ergeben.}
        \newline
        Somit existieren zwei verschiedene $M$-Tupel $\zeta_1, \zeta_2$, sodass gilt:
        \begin{equation*}
            L_j(\zeta_1) = L_j(\zeta_2) \sptext{mit} 1 \leq j \leq M.
        \end{equation*}
        mit $\zeta = \zeta_1 - \zeta_2$ folgt die Behauptung.
    \end{proof}

\subsection{Lemma 4}
    \label{subsec:lemma4}
    \textrm{Sei $\xi$ algebraisch irrational und $n$ dessen Grad. Für alle $l \in \mathbb{N}_0$ existieren
    Koeffizienten $a_{j,l} \in \mathbb{Z}$ mit $1 \leq j < n$, sodass gilt:}
    \begin{equation*}
        \xi^l = \dsum{a_{n-1, l}\xi^{n-1}}{a_{0,l}}
    \end{equation*}
    \textrm{und sind mit $a$ aus~\eqref{eq:def-a} folgendermaßen nach oben beschränkt:}
    \begin{equation*}
        \left| a_{a,l} \right| \leq (a + 1)^l
    \end{equation*}
    \begin{proof}
        \textrm{Für $l < n$ ist die Aussage offensichtlich. Für $l \geq n$ wird eine kurze Induktion aufgezogen:}
        \newpage
        Induktionsanfang: $l = n$
        \begin{indentpar}
            Folgt durch Umformung aus $f(x) = 0$ aus~\ref{subsec:algebraic-numbers}.
        \end{indentpar}
        Induktionsvoraussetzung:
        \begin{indentpar}
            \textrm{Angenommen, die Behauptung gilt für ein $l > n$.}
        \end{indentpar}
        Induktionsschritt: $l \Rightarrow l+1$
        \begin{indentpar}
            Folgt mit Hilfe von
            \begin{equation*}
                \xi^{l+1} = \xi \cdot \xi^l = \dsum{a_{n-1, l}\xi^{n}}{a_{0,l} \xi}.
            \end{equation*}
        \end{indentpar}
        \textrm{Per Induktionsaxiom folgt die Aussage für alle $l \in \mathbb{N}$.}
    \end{proof}

\subsection{Lemma 5}
    \label{subsec:lemma5}
    \textrm{Seien $\enum{r_1}{r_m} \in \mathbb{N}$ und $0 < \lambda \in \mathbb{R}$.\ Dann gibt es höchstens}
    \begin{equation*}
        \sqrt{2m} \lambda^{-1} \spdots{(r_1 + 1)}{(r_m +1)}
    \end{equation*}
    \textrm{viele $m$-Tupel $(\enum{i_1}{i_m}) \in \mathbb{Z}^m$, die die folgende Ungleichung erfüllen:}
    \begin{equation*}
        \sum_{\substack{0 \leq i_\mu \leq r_\mu \\ 1 \leq \mu \leq m}} \frac{i_\mu}{r_\mu} \leq \den{2} (m - \lambda)
    \end{equation*}
    \begin{proof}
        \textrm{Falls $m = 1$, sieht man leicht, dass höchstens $r_1 + 1$ viele Lösungen möglich sind und die Ungleichung
        sogar unlösbar ist, falls $\lambda > 1$.
        \newline
        Für $\sqrt{2m} > \lambda$ ist das Lemma ebenfalls trivial.}
        \newline
        \textrm{Angenommen, die Aussage gelte für $m-1$ und seien nun}
        \begin{equation}
            m > 1 \sptext{und} \lambda > \sqrt{2m} > 1. \label{eq:sqrt-2m-ass}
        \end{equation}
        \textrm{Seien $r = r_m$ und $i = i_m$ beliebig, aber fest.\ Dann beschränkt sich die Anzahl der ganzen Zahlen
        $\enum{i_1}{i_{m-1}}$ auf höchstens}
        \begin{equation}
            \sqrt{2m - 2} \left( \lambda - 1 + 2 \frac{i}{r} \right)^{-1}\enum{(r_1+1)}{(r_{m-1}+1)}. \label{eq:fix-lim}
        \end{equation}
        Man sieht schnell:
        \begin{align*}
            \sum_{0 \leq i \leq r} \frac{2}{\lambda - 1 + 2\frac{i}{r}} &= \sum_{0 \leq i \leq r} \left( \den{\lambda -
            1 + 2 \frac{i}{r}} + \den{\lambda + 1 - 2 \frac{i}{r}} \right) \\ &= \sum_{0 \leq i \leq r} \frac{2 \lambda}
            {\lambda^2 - \left( 1 - 2 \frac{i}{r} \right)^2} < \frac{2 (r + 1) \lambda}{\lambda^2 - 1}
        \end{align*}
        \textrm{und durch die Beschränkung aus~\ref{eq:fix-lim} folgt:}
        \begin{equation*}
            \lambda^2 - 1 > \lambda^2 (1 - (2m)^{-1}) > \lambda^2 \sqrt{1 - m^{-1}}.
        \end{equation*}
        \textrm{Für $i = i_m \in (\enum{0}{r = r_m})$ folgt nun die Behauptung.}
    \end{proof}
    

    
    \subsection{Beweis des \emph{Theorem II}}
        \label{subsec:proof-th2}
        \textrm{Wie in der \hyperref[sec:th2]{Einleitung des Kapitels} erwähnt, läuft der Beweis darauf hinaus, das
        \hyperref[subsec:lemma3]{Lemma~3} anzuwenden.\ Die nötige Vorarbeit wird in erster Linie mit den Lemmata
        \hyperref[subsec:lemma4]{4} und \hyperref[subsec:lemma5]{5} geleistet.}
        \begin{namedproof}{\emph{\hyperref[subsec:th2]{Theorem II}}}
            Sei $R$ ein Polynom nach Definition in~\ref{subsubsec:def-poly} und seien $\varepsilon,\ \xi,\ n,\ m \sptext
            {und} \enum{r_1}{r_m}$ wie in~\ref{subsec:th2} beschrieben.
            \newline
            Zu zeigen ist, dass die $\spdots{(r_1+1)}{(r_m+1)} = N$ vielen Koeffizienten von R existieren.
            \newline 
            Dazu soll gelten $\forall \ \enum{i_1}{i_m} \in \mathbb{N}$:
            \begin{equation}
                R_{\enum{i_1}{i_m}}(\enum{\xi}{\xi}) = 0 \label{eq:rem-poly-condition}
            \end{equation}
            sodass ebenfalls gilt:
            \begin{equation}
                \sum_{1 \leq \mu \leq m} \frac{i_\mu}{r_\mu} \leq \den{2} m(1 - \varepsilon) \label{eq:ind-condition}
            \end{equation}
            Da das Polynom verschwindet und~\eqref{eq:rem-poly-condition} somit trivial ist, falls $i_\mu > r_\mu$, sei
            also $i_\mu \leq r_\mu \ (\spaceoperator 1 \leq \mu \leq m)$.
            \newline
            \hyperref[subsec:lemma4]{Lemma~4} erlaubt es, alle Potenzen von $\xi$ als Linearkombinationen von $1, \enum
            {\xi}{\xi^{n-1}}$ zu schreiben.\ Diese lassen sich als lineares Gleichungssystem mit $n$ Gleichungen in den
            Variablen $c_{\enum{j_1}{j_m}}$ auffassen:
            \begin{equation*}
                L_k = \sum_{\substack{0 \leq j_\mu \leq r_\mu \\ 1 \leq \mu \leq m}} a_{k, \enum{j_1}{j_m}} c_{\enum{j_1}
                {j_m}} \sptext{mit} 1 \leq k \leq n.
            \end{equation*}
            \hyperref[subsec:lemma4]{Lemma~4} und~\eqref{eq:2.4} liefern, dass die Koeffizienten die folgende Form haben:
            \begin{equation*}
                \spdots{\binom{j_1}{i_1}}{\binom{j_m}{i_m}}a_{j,l} \sptext{mit} 0 \leq j < n
            \end{equation*}
            wobei $l = \dsum{(j_1 - i_1)}{(j_m - i_m)} \leq \dsum{r_1}{r_m}$ sei.
            \textrm{Wegen~\eqref{eq:2.5} und erneut \hyperref[subsec:lemma4]{Lemma~4} sind die Koeffizienten betragsmäßig
            nach oben beschränkt durch:}
            \begin{equation}
                A = (2a + 2)^{\dsum{r_1}{r_m}} \label{eq:coeff-condition}
            \end{equation}
            \textrm{Das \hyperref[subsec:lemma5]{Lemma~5} (mit $\lambda = m \varepsilon$) und~\eqref{eq:def-m} liefern,
            dass die Anzahl $M$ an Gleichungen im System $L$ in $N$ Variablen aus~\hyperref[subsec:lemma3]{Lemma~3}
            folgendermaßen beschränkt sind:}
            \begin{equation}
                M \leq n \frac{\sqrt{2m}}{m \varepsilon} N \leq \den{2} N \label{eq:M-condition}
            \end{equation}
            Dies zeigt~\ref{st:ind-st}.
            \newline
            \textrm{Nun sagt \hyperref[subsec:lemma3]{Lemma~3} aus, dass die Koeffizienten $c_{\enum{j_1}{j_m}}$ als ganze
            Zahlen existieren, nicht alle veschwinden (was~\ref{st:R-st} impliziert) und~\eqref{eq:rem-poly-condition},
            ~\eqref{eq:ind-condition} und die Beschränkung $i_\mu~\leq~r_\mu$ erfüllen und wegen~\eqref{eq:coeff-condition}
            und~\eqref{eq:M-condition} betrangsmäßig beschränkt sind durch:}
            \begin{equation*}
                \left| c_{\enum{j_1}{j_m}} \right| \leq (NA)^{\frac{M}{N-M}} \leq NA \leq \gamma^{\dsum{r_1}{r_m}}
            \end{equation*}
            und mit~\eqref{eq:2.5} folgt
            \begin{equation*}
                N = \spdots{(r_1 + 1)}{(r_m + 1)} \leq \gamma^{\dsum{r_1}{r_m}}.
            \end{equation*}
            Dies zeigt~\ref{st:cont-st}; somit gelten alle Aussagen aus \hyperref[subsec:th2]{\emph{Theorem~II}}.
        \end{namedproof}
    

        
\end{document}