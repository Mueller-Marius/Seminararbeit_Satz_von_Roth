
\section{Grundlagen und Voraussetzungen}
    \label{sec:preliminaries}
    \textrm{In diesem Kapitel werden die Grundlagen behandelt, die für den Beweis des \emph{Satzes von Roth} benötigt
    werden. Außerdem wird die formale Aussage des Satzes dargelegt; der Beweis wird jedoch nicht in dieser Arbeit vollzogen.}
    
    \subsection{Notation}
    \label{subsec:notation}
        \textrm{In dieser Arbeit wird $\mathbb{N}$ verwendet als natürliche Zahlen \emph{ohne} Null, wogegen $\mathbb{N}_0$
        die natürlichen Zahlen \emph{inklusive} der Null meint.}
    
    
\subsection{algebraische Zahlen}
    \label{subsec:algebraic-numbers}
    \textrm{Eine komplexe Zahl $z \in \mathbb{C}$ heißt \emph{algebraisch} genau dann, wenn gilt:}
    \begin{equation}
        \exists f \in \mathbb{Q}[x] : f(z) = 0 \label{eq:def-algebraic}
    \end{equation}
    \textrm{d.h.\ falls z eine Lösung eines Polynoms mit rationalen Koeffizienten $a_k~=~\frac{p_k}{q_k}$ mit $p_k \in
    \mathbb{Z},\ q_k \in \mathbb{N}\ (\forall \ 1 \leq k \leq n)$ ist.
    \newline
    Die Zahl $n = \deg f$ heißt \emph{Grad} der algebraischen Zahl $z$.
    \newline
    OBdA kann angenommen werden, dass $f \in \mathbb{Z}[x]$, da sich die Gleichung $f(x) = 0$ mit dem Produkt der Nenner
    der Koeffizienten multiplizieren lässt, wodurch alle Koeffizienten ganzzahlig werden, die Gleichung und damit auch
    das resultierende Polynom jedoch dieselben Lösungen bzw.\ Nullstellen besitzen.\ Weiterhin lässt sich oBdA annehmen,
    dass für den Koeffizienten der höchsten Potenz von $x$ gilt: $a_n \neq 0$.}

\subsection{Der Satz von Roth (\emph{Theorem I})}
    \label{subsec:th1}
    Sei $\xi \in \mathbb{R}$ algebraisch irrational und $\delta > 0$ beliebig.\ Dann besitzt die Ungleichung
    \begin{equation}
        0 < \left| \xi - \frac{p}{q} \right| < q^{-(2+\delta)} \label{eq:svr}
    \end{equation}
    \textrm{nur endlich viele Lösungen in $p \in \mathbb{Z}$, $q \in \mathbb{N}$.
    \newpage
    \textrm{Hiermit liefert der Satz die obere Schranke $2$ für das Irrationalitätsmaß algebraisch irrationaler Zahlen;
    zusammen mit der unteren Schranke von ebenfalls $2$ gilt somit $\mu(x) = 2$ für alle algebraisch irrationalen Zahlen
    $x$.}
    \newline \newline
    Zunächst wird gezeigt, dass der Satz nur für $a_n = 1$ aus~\ref{subsec:algebraic-numbers} zu zeigen ist.}

\subsection{Normieren des Polynoms einer algebraischen Zahl}
    \label{subsec:norm-poly}
    \textrm{Angenommen, der \hyperref[eq:svr]{\emph{Satz von Roth}} gelte.\ Dann folgt aus $f(\xi) = 0$ aus~\ref
    {subsec:algebraic-numbers} durch Multiplikation mit $a_n^{n-1}$ für $a_n \xi \coloneq \Xi$:}
    \begin{equation*}
        0 = \Xi^n + a_{n-1} \Xi^{n-1} + \dsum{a_{n-2}a_n \Xi^{n-2}}{a_n^{n-1} a_0}
    \end{equation*}
    \textrm{und nach Multiplikation mit $\left| a_n \right|$ und geeigneter Abschätzung von~\eqref{eq:svr} gilt:}
    \begin{equation*}
        \left| \Xi - a_n \frac{p}{q} \right| < \left| a_n \right| q^{-(2+\delta)} < q^{-(2+\den{2}\delta)}
    \end{equation*}
    \textrm{für hinreichend große $q$.\ Da $\delta$ beliebig gewählt wurde, gilt der Satz somit nun für $\Xi$ genau dann,
    wenn er für $\xi$ gilt. Somit gilt insgesamt oBdA:}
    \begin{equation}
        f(x) = x^n + \dsum{a_{n-1} x^{n-1}}{a_0} \sptext{mit} f(\xi) = 0 \sptext{und} \enum{a_{n-1}}{a_0}
        \in \mathbb{Z}. \label{eq:prereq}
    \end{equation}
    \newline
    \textrm{Sei im Übrigen}
    \begin{equation}
        a = \max\{ 1, \enum{|a_{n-1}|}{|a_0|} \}. \label{eq:def-a}
    \end{equation}
    \newline
    Dies wird im weiteren Verlauf der Arbeit und des Beweises verwendet.
    

    
    
\subsection{Polynome}
    \label{subsec:polynomials}
    \textrm{In diesem Abschnitt werden Polynome und weitere Begriffe definiert, die in diesem Kapitel verwendet werden.
    Darauf folgen zwei Lemmata, die diese jeweils kurz näher beleuchten.}
    
    \subsubsection{Definition der Polynome}
        \label{subsubsec:def-poly}
        Es werden Polynome der folgenden Form behandelt:
        \begin{equation}
            R : \mathbb{R}^m \rightarrow \mathbb{R},\
            R(\enum{x_1}{x_m}) = \sum_{\substack{0 \leq j_\mu \leq r_\mu \\ 1 \leq \mu \leq m}}
            c_{\enum{j_1}{j_m}} \cdot \spdots{x_1^{j_1}}{x_m^{j_m}} \label{eq:def-poly}
        \end{equation}
        wobei $m \in \mathbb{N}$, $r_\mu$ der Grad des Polynoms in $x_\mu$ und $c_{\enum{j_1}{j_m}} \in \mathbb{R}$ seien.
        \newpage
        \emph{Beispiel}: Seien $m = 2,\ r_1 = 2 \sptext{und} r_2 = 1$.
        \newline
        \textrm{Das zugehörige Polynom nach obiger Definition sieht folgendermaßen aus:}
        \begin{equation*}
            R_{Bsp}(x_1, x_2) = c_{0,0} x_1^0 x_2^0 + c_{0,1} x_1^0 x_2^1 + c_{1,0} x_1^1 x_2^0 + c_{1,1} x_1^1 x_2^1 +
            c_{2, 0} x_1^2 x_2^0 + c_{2,1} x_1^2 x_2^1
        \end{equation*}
        Nach konkreter Definition der Koeffizienten ergibt sich beispielsweise:
        \begin{equation*}
            R_{Bsp}(x_1, x_2) = 7 - \sqrt{\pi} x_2^1 + \frac{e}{\cbrt{\gamma}} x_1^2 x_2^1
        \end{equation*}
    
    \subsubsection{Der Inhalt eines Polynoms}
        \label{subsubsec:def-content}
        Sei R ein Polynom nach \hyperref[subsubsec:def-poly]{obiger Definition}.\ Dann sei der \emph{Inhalt} eines
        Polynoms wie folgt definiert:
        \begin{equation}
            \content{R} \coloneq \max\{|c_{\enum{j_1}{j_m}}| \} \  \forall \  0 \leq j_\mu \leq r_\mu, \
            1 \leq \mu \leq m \label{eq:def-content}
        \end{equation}
    
    \subsubsection{Das Restpolynom}
        \label{subsubsec:def-remainder-poly}
        Sei R ein Polynom nach \hyperref[subsubsec:def-poly]{obiger Definition} und seien $\enum{i_1}{i_m} \in \mathbb
        {N}_0$.\ Es wird
        \begin{equation}
            R_{\enum{i_1}{i_m}} = \den{\enum{i_1!}{i_m!}} \frac{\partial^{\dsum{i_1}{i_m}}}{
                \spdots{\partial x_1^{i_1}}{\partial x_m^{i_m}}} (R) \label{eq:def-rempoly}
        \end{equation}
        das \emph{Restpolynom} von R genannt.
    
    \subsubsection{Der Index eines Polynoms}
        \label{subsubsec:def-index}
        \textrm{Sei R ein Polynom nach \hyperref[subsubsec:def-poly]{obiger Definition}, $\alpha \in \mathbb{R}^m$ und
            $s \in \mathbb{N}^m$.\ $I \in \mathbb{R}$ heißt \emph{Index} von R an der Stelle $\alpha$ bezüglich $s$,
            genau dann, wenn gilt:}
        \begin{equation}
            I \coloneq \ind(R) \coloneq \min_{(\enum{i_1}{i_m}) \in \mathbb{N}_0^m} \sum_{1 \leq \mu \leq m} \frac{i_\mu}
            {s_\mu}, \sptext{sodass} R_{i_1, \dots, i_m}(\alpha) \neq 0. \label{eq:def-index}
        \end{equation}
        Falls R verschwindet, wird konventionell $\ind(R) \coloneq \infty$ gesetzt.
        \newline
        \textrm{Dies erinnert an die \emph{Nullstellenordnung} einer Funktion (beispielsweise Nullstellen \emph
        {zweiter Ordnung} beziehungsweise \emph{zweifache} Nullstellen), lässt sich jedoch mit Hilfe des $m$-Tupels $s$
        noch zusätzlich gewichten.
        \newline
        Zu diesem Begriff folgt wie erwähnt ein \hyperref[subsec:lemma2]{Lemma}, das weitere Eigenschaften darlegt.\ In
        dieser Arbeit wird diese Definition über einen Nebensatz hinaus jedoch nicht weiter verwendet; er findet lediglich
            in den Beweisen der weiteren Theorems im Beweis des \emph{Satzes von Roth} weitere Anwendung.}
    

    
    \subsection{Lemma 1}
        \label{subsec:lemma1}
        Sei R ein Polynom nach \hyperref[subsubsec:def-poly]{obiger Definition}.\ Dann gilt:
        \begin{enumerate}
            \item \textrm{$R$ hat ausschließlich Koeffizienten in $\mathbb{Z} \Rightarrow R_{\enum{i_1}{i_m}}$ hat
            auch ausschließlich Koeffizienten in $\mathbb{Z}$}
            \item \textrm{$R$ hat Grad $r_\mu$ in $x_\mu \Rightarrow R_{\enum{i_1}{i_m}}$ hat höchstens Grad $r_\mu
            - i_\mu$ in $x_\mu$ und verschwindet für $r_\mu < i_\mu$ ($\forall \  1 \leq \mu \leq m$)}
            \item $\content{R_{\enum{i_1}{i_m}}} \leq 2^{\dsum{r_1}{r_m}} \content{R}$
        \end{enumerate}
        \textrm{Diese Aussagen sind recht trivial; der Beweis wird nur vollzogen, da die folgenden Aussagen im späteren
        Verlauf benötigt werden.}
        \begin{proof}
            Die Aussagen folgen aus
            \begin{equation}
                R_{\enum{i_1}{i_m}}(\enum{x_1}{x_m}) = \sum_{\substack{0 \leq j_\mu \leq r_\mu \\ 1 \leq \mu \leq m}}
                \spdots{\binom{j_1}{i_1}}{\binom{j_m}{i_m}}c_{\enum{j_1}{j_m}} \cdot \spdots{x_1^{j_1}}{x_m^{j_m}},
                \label{eq:2.4}
            \end{equation}
            \textrm{wobei die Binominalkoeffizienten $\binom{j}{i}$ nach oben beschränkt sind durch:}
            \begin{equation}
                \binom{j}{i} \leq \sum_{0 \leq i \leq j} \binom{j}{i} = (1 + 1)^j \leq 2^r. \label{eq:2.5}
            \end{equation}
        \end{proof}
    
    \subsection{Lemma 2}
        \label{subsec:lemma2}
        Seien $\alpha \in \mathbb{R}^m,\ s \in \mathbb{N}^m$ und $R$ und $T$ Polynome nach \hyperref
        [subsubsec:def-poly]{obiger Definition}.\ Dann gilt:
        \begin{enumerate}
            \item $\ind(R_{\enum{i_1}{i_m}}) \geq \ind(R) - \sum \frac{i_\mu}{s_\mu}$
            \item $\ind(R_1 + R_2) \geq \min \{ \ind(R_1), \ind(R_2) \}$
            \item $\ind(R \cdot T) = \ind(R) +\ind(T)$
        \end{enumerate}
        \textrm{Da der Begriff des \emph{Index} in dieser Arbeit keine weitere Anwendung findet und die Aussagen
        außerdem nach kurzem Durchdenken ebenfalls recht trivial sind, wird auch hier auf einen ausführlichen Beweis
        verzichtet.}

    
