
\section{Konstruktion des Polynoms R (\emph{Theorem II})}
    \label{sec:th2}
    \textrm{In diesem Kapitel wird das Polynom $R$ konstruiert, das später im Beweis des \emph{Satzes von Roth}
    verwendet werden wird.\ Zum Beweis des \emph{Theorem II} werden drei Lemmata benötigt, von denen das \hyperref
    [subsec:lemma3]{Lemma 3} das zentrale Lemma ist, das am Ende die Bestimmung der Koeffizienten des Polynoms
    ermöglicht.}
    
    \subsection{Aussage des \emph{Theorem II}}
        \label{subsec:th2}
        Sei $\varepsilon > 0$ beliebig, $\xi$ algebraisch irrational, $n$ der Grad von $\xi$, sei
        \begin{equation}
            m \in \mathbb{Z} \text{, sodass} \  m > 8 n^2 \varepsilon^{-2} \label{eq:def-m}
        \end{equation}
        \textrm{und seien $\enum{r_1}{r_m} \in \mathbb{N}$.
        Dann existiert ein $R$ nach Definition in~\ref{subsubsec:def-poly} mit ganzzahligen Koeffizienten und höchstens
        Grad $r_\mu$ in $x_\mu \ (\forall \ 1 \leq \mu \leq m)$, für das gilt:}
        \begin{enumerate}
            \item $R$ verschwindet nicht
            \item \textrm{$\ind(R) \geq \den{2} m (1 - \varepsilon)$ an der Stelle $(\enum{\xi}{\xi})$ bezüglich
                $(\enum{r_1}{r_m})$}
            \item $\content{R} \leq \gamma^{\dsum{r_1}{r_m}}$ mit $\gamma = 4 (a + 1)$
        \end{enumerate}
        \textrm{mit a aus~\eqref{eq:def-a}.\ Hier ist vor allem wichtig, dass diese Aussagen für ein $m >$ eine Konstante
        abhängig von $\xi$ und $\varepsilon$ gelten.\ Der genaue Wert von $\gamma$ ist ebenfalls recht irrelevant, er
        kann von $\varepsilon$, $\xi$ oder $m$ abhängen.}
        \newline \newline
        \textrm{Die folgenden drei Lemmata werden für den \hyperref[subsec:proof-th2]{Beweis} benötigt:}
    
    
\subsection{Lemma 3}
    \label{subsec:lemma3}
    Seien $N, M \in \mathbb{N} \sptext{mit} N > M$ und seien
    \begin{equation*}
        L_j(\enum{z_1}{z_N}) = \sum_{1 \leq k \leq N} a_{jk} z_k \sptext{mit} 1 \leq j \leq M
    \end{equation*}
    $M$ viele $N$-Linearformen mit Koeffizienten $a_k \in \mathbb{Z}$ in $N$ vielen Variablen $z_k$.
    \textrm{Sei außerdem} $A \in \mathbb{N}$, sodass
    \begin{equation*}
        \left| a_{jk} \right| \leq A \ \ \forall \ 1 \leq j \leq M, 1 \leq k \leq N
    \end{equation*}
    \textrm{Dann besitzt das System $L = (\enum{L_1}{L_M})$ in den Variablen $\enum{z_1}{z_N}$ Lösungen in $\mathbb{Z}$,
    die nicht alle verschwinden und sodass gilt:}
    \begin{equation*}
        L_j = 0 \sptext{mit} 1 \leq j \leq M \sptext{und} \left| z_i \right| \leq Z = \left[ (NA)^{\frac{M}{N-M}} \right]
        \sptext{mit} 1 \leq k \leq n.
    \end{equation*}
    \begin{proof}
        Seien $N, M, A, Z$ wie oben beschrieben.\ Sei $- B_j$ die Summe der negativen und $P_j$ die Summe der positiven
        Koeffizienten in $L_j(z)$.
        \newline
        Es gilt:
        \begin{equation*}
            NA < (Z + 1)^{\frac{N-M}{M}}
        \end{equation*}
        woraus folgt:
        \begin{equation*}
            NAZ+1 \leq NA(Z + 1) < (Z + 1)^{\frac{N}{M}}
        \end{equation*}
        \textrm{Für alle} $N$-Tupel $\zeta = (\enum{z_1}{z_N})$ mit
        \begin{equation}
            0 \leq z_k \leq Z \sptext{mit} 1 \leq k \leq N \label{eq:z_k-condition}
        \end{equation}
        gilt:
        \begin{equation*}
            - B_j Z \leq L_j(z) \leq P_j Z \sptext{und} B_j + P_j \leq N A
        \end{equation*}
        \textrm{Somit ergeben sich für $L_j(z) \in \mathbb{Z}$ höchstens $NAZ+1$ viele verschiedene Werte. Es existieren
        auch höchstens $(Z + 1)^N$ viele verschiedene $\zeta$, woraus sich jedoch höchstens $(Z + 1)^N > (NAZ+1)^M$ viele
        verschiedene Systeme $L$ ergeben.}
        \newline
        Somit existieren zwei verschiedene $M$-Tupel $\zeta_1, \zeta_2$, sodass gilt:
        \begin{equation*}
            L_j(\zeta_1) = L_j(\zeta_2) \sptext{mit} 1 \leq j \leq M.
        \end{equation*}
        mit $\zeta = \zeta_1 - \zeta_2$ folgt die Behauptung.
    \end{proof}

\subsection{Lemma 4}
    \label{subsec:lemma4}
    \textrm{Sei $\xi$ algebraisch irrational und $n$ dessen Grad. Für alle $l \in \mathbb{N}_0$ existieren
    Koeffizienten $a_{j,l} \in \mathbb{Z}$ mit $1 \leq j < n$, sodass gilt:}
    \begin{equation*}
        \xi^l = \dsum{a_{n-1, l}\xi^{n-1}}{a_{0,l}}
    \end{equation*}
    \textrm{und sind mit $a$ aus~\eqref{eq:def-a} folgendermaßen nach oben beschränkt:}
    \begin{equation*}
        \left| a_{a,l} \right| \leq (a + 1)^l
    \end{equation*}
    \begin{proof}
        \textrm{Für $l < n$ ist die Aussage offensichtlich. Für $l \geq n$ wird eine kurze Induktion aufgezogen:}
        \newpage
        Induktionsanfang: $l = n$
        \begin{indentpar}
            Folgt durch Umformung aus $f(x) = 0$ aus~\ref{subsec:algebraic-numbers}.
        \end{indentpar}
        Induktionsvoraussetzung:
        \begin{indentpar}
            \textrm{Angenommen, die Behauptung gilt für ein $l > n$.}
        \end{indentpar}
        Induktionsschritt: $l \Rightarrow l+1$
        \begin{indentpar}
            Folgt mit Hilfe von
            \begin{equation*}
                \xi^{l+1} = \xi \cdot \xi^l = \dsum{a_{n-1, l}\xi^{n}}{a_{0,l} \xi}.
            \end{equation*}
        \end{indentpar}
        \textrm{Per Induktionsaxiom folgt die Aussage für alle $l \in \mathbb{N}$.}
    \end{proof}

\subsection{Lemma 5}
    \label{subsec:lemma5}
    \textrm{Seien $\enum{r_1}{r_m} \in \mathbb{N}$ und $0 < \lambda \in \mathbb{R}$.\ Dann gibt es höchstens}
    \begin{equation*}
        \sqrt{2m} \lambda^{-1} \spdots{(r_1 + 1)}{(r_m +1)}
    \end{equation*}
    \textrm{viele $m$-Tupel $(\enum{i_1}{i_m}) \in \mathbb{Z}^m$, die die folgende Ungleichung erfüllen:}
    \begin{equation*}
        \sum_{\substack{0 \leq i_\mu \leq r_\mu \\ 1 \leq \mu \leq m}} \frac{i_\mu}{r_\mu} \leq \den{2} (m - \lambda)
    \end{equation*}
    \begin{proof}
        \textrm{Falls $m = 1$, sieht man leicht, dass höchstens $r_1 + 1$ viele Lösungen möglich sind und die Ungleichung
        sogar unlösbar ist, falls $\lambda > 1$.
        \newline
        Für $\sqrt{2m} > \lambda$ ist das Lemma ebenfalls trivial.}
        \newline
        \textrm{Angenommen, die Aussage gelte für $m-1$ und seien nun}
        \begin{equation}
            m > 1 \sptext{und} \lambda > \sqrt{2m} > 1. \label{eq:sqrt-2m-ass}
        \end{equation}
        \textrm{Seien $r = r_m$ und $i = i_m$ beliebig, aber fest.\ Dann beschränkt sich die Anzahl der ganzen Zahlen
        $\enum{i_1}{i_{m-1}}$ auf höchstens}
        \begin{equation}
            \sqrt{2m - 2} \left( \lambda - 1 + 2 \frac{i}{r} \right)^{-1}\enum{(r_1+1)}{(r_{m-1}+1)}. \label{eq:fix-lim}
        \end{equation}
        Man sieht schnell:
        \begin{align*}
            \sum_{0 \leq i \leq r} \frac{2}{\lambda - 1 + 2\frac{i}{r}} &= \sum_{0 \leq i \leq r} \left( \den{\lambda -
            1 + 2 \frac{i}{r}} + \den{\lambda + 1 - 2 \frac{i}{r}} \right) \\ &= \sum_{0 \leq i \leq r} \frac{2 \lambda}
            {\lambda^2 - \left( 1 - 2 \frac{i}{r} \right)^2} < \frac{2 (r + 1) \lambda}{\lambda^2 - 1}
        \end{align*}
        \textrm{und durch die Beschränkung aus~\ref{eq:fix-lim} folgt:}
        \begin{equation*}
            \lambda^2 - 1 > \lambda^2 (1 - (2m)^{-1}) > \lambda^2 \sqrt{1 - m^{-1}}.
        \end{equation*}
        \textrm{Für $i = i_m \in (\enum{0}{r = r_m})$ folgt nun die Behauptung.}
    \end{proof}
    

    
    \subsection{Beweis des \emph{Theorem II}}
        \label{subsec:proof-th2}
        \textrm{Wie in der \hyperref[sec:th2]{Einleitung des Kapitels} erwähnt, läuft der Beweis darauf hinaus, das
        \hyperref[subsec:lemma3]{Lemma 3} anzuwenden.\ Die nötige Vorarbeit wird in erster Linie mit den Lemmata
        \hyperref[subsec:lemma4]{4} und \hyperref[subsec:lemma5]{5} geleistet.}
        \begin{namedproof}{\emph{\hyperref[subsec:th2]{Theorem II}}}
            Sei $R$ ein Polynom nach Definition in~\ref{subsubsec:def-poly} und seien $\varepsilon,\ \xi,\ n,\ m \sptext
            {und} \enum{r_1}{r_m}$ wie in~\ref{subsec:th2} beschrieben.
            \newline
            Zu zeigen ist, dass die $\spdots{(r_1+1)}{(r_m+1)} = N$ vielen Koeffizienten von R existieren.
            \newline 
            Dazu soll gelten $\forall \ \enum{i_1}{i_m} \in \mathbb{N}$:
            \begin{equation}
                R_{\enum{i_1}{i_m}}(\enum{\xi}{\xi}) = 0 \label{eq:rem-poly-condition}
            \end{equation}
            sodass ebenfalls gilt:
            \begin{equation*}
                \sum_{1 \leq \mu \leq m} \frac{i_\mu}{r_\mu} \leq \den{2} m(1 - \varepsilon)
            \end{equation*}
            Da das Polynom verschwindet und~\eqref{eq:rem-poly-condition} somit trivial ist, falls $i_\mu > r_\mu$, sei
            also $i_\mu \leq r_\mu$ ($\forall \ 1 \leq \mu \leq m$).
            
        \end{namedproof}
    
