
\subsection{Lemma 3}
    \label{subsec:lemma3}
    Seien $N, M \in \mathbb{N} \sptext{mit} N > M$ und seien
    \begin{equation*}
        L_j = \sum_{1 \leq k \leq N} a_{jk} z_k \sptext{mit} 1 \leq j \leq M
    \end{equation*}
    $M$ viele $N$-Linearformen mit Koeffizienten $a_k \in \mathbb{Z}$ in $N$ vielen Variablen $z_k$.
    \textrm{Sei außerdem} $A \in \mathbb{N}$, sodass
    \begin{equation*}
        \left| a_{jk} \right| \leq A \ \ \forall \ 1 \leq j \leq M, 1 \leq k \leq N
    \end{equation*}
    \textrm{Dann besitzt das System $L$ in den Variablen $\enum{z_1}{z_N}$ Lösungen in $\mathbb{Z}$, die nicht alle
    verschwinden und sodass gilt:}
    \begin{equation*}
        L_j = 0 \sptext{mit} 1 \leq j \leq M \sptext{und} \left| z_i \right| \leq Z = \left[ (NA)^{\frac{M}{N-M}} \right]
        \sptext{mit} 1 \leq k \leq n.
    \end{equation*}
    \begin{proof}
        bla
    \end{proof}

\subsection{Lemma 4}
    \label{subsec:lemma4}
    \textrm{Sei $\xi$ algebraisch irrational und $n$ dessen Grad. Für alle $l \in \mathbb{N}_0$ existieren
    Koeffizienten $a_{j,l} \in \mathbb{Z}$ mit $1 \leq j < n$, sodass gilt:}
    \begin{equation*}
        \xi^l = \dsum{a_{n-1, l}\xi^{n-1}}{a_{0,l}}
    \end{equation*}
    \textrm{und sind mit $a$ aus~\eqref{eq:def-a} folgendermaßen nach oben beschränkt:}
    \begin{equation*}
        \left| a_{a,l} \right| \leq (a + 1)^l
    \end{equation*}
    \begin{proof}
        \textrm{Für $l < n$ ist die Aussage offensichtlich. Für $l \geq n$ wird eine kurze Induktion aufgezogen:}
        \newpage
        Induktionsanfang: $l = n$
        \begin{indentpar}
            Folgt durch Umformung aus $f(x) = 0$ aus~\ref{subsec:algebraic-numbers}.
        \end{indentpar}
        Induktionsvoraussetzung:
        \begin{indentpar}
            \textrm{Angenommen, die Behauptung gilt für ein $l > n$.}
        \end{indentpar}
        Induktionsschritt: $l \Rightarrow l+1$
        \begin{indentpar}
            Folgt mit Hilfe von
            \begin{equation*}
                \xi^{l+1} = \xi \cdot \xi^l = \dsum{a_{n-1, l}\xi^{n}}{a_{0,l} \xi}.
            \end{equation*}
        \end{indentpar}
        \textrm{Somit gilt die Aussage für alle $l \in \mathbb{N}$.}
    \end{proof}

\subsection{Lemma 5}
    \label{subsec:lemma5}
    \textrm{Seien $\enum{r_1}{r_m} \in \mathbb{N}$ und $0 < \lambda \in \mathbb{R}$.\ Dann gibt es höchstens}
    \begin{equation*}
        \frac{\sqrt{2m}}{\lambda} \spdots{(r_1 + 1)}{(r_m +1)}
    \end{equation*}
    \textrm{viele $m$-Tupel $(\enum{i_1}{i_m})$, die die folgende Ungleichung erfüllen:}
    \begin{equation*}
        \sum_{\substack{0 \leq i_\mu \leq r_\mu \\ 1 \leq \mu \leq m}} \frac{i_\mu}{r_\mu} \leq \den{2} (m - \lambda)
    \end{equation*}
    \begin{proof}
        bla \shine
    \end{proof}
    
