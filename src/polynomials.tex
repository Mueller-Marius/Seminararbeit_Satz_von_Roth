
\subsection{Polynome}
    \label{subsec:polynomials}
    \textrm{In diesem Abschnitt werden Polynome und weitere Begriffe definiert, die in diesem Kapitel verwendet werden.
    Darauf folgen zwei Lemmata, die diese jeweils kurz näher beleuchten.}
    
    \subsubsection{Definition der Polynome}
        \label{subsubsec:def-poly}
        Es werden Polynome der folgenden Form behandelt:
        \begin{equation}
            R : \mathbb{R}^m \rightarrow \mathbb{R},\
            R(\enum{x_1}{x_m}) = \sum_{\substack{0 \leq j_\mu \leq r_\mu \\ 1 \leq \mu \leq m}}
            c_{\enum{j_1}{j_m}} \cdot \spdots{x_1^{j_1}}{x_m^{j_m}} \label{eq:def-poly}
        \end{equation}
        Wobei $m \in \mathbb{N}$, $r_\mu$ der Grad des Polynoms in $x_\mu$ und $c_{j_1, \dots, j_m} \in
        \mathbb{R}$ und seien.
        \newline \newline
        \emph{Beispiel}: Seien $m = 2,\ r_1 = 2 \sptext{und} r_2 = 1$.
        \newline
        \textrm{Das zugehörige Polynom nach obiger Definition sieht folgendermaßen aus:}
        \begin{equation*}
            R_{Bsp}(x_1, x_2) = c_{0,0} x_1^0 x_2^0 + c_{0,1} x_1^0 x_2^1 + c_{1,0} x_1^1 x_2^0 + c_{1,1} x_1^1
            x_2^1 + c_{2, 0} x_1^2 x_2^0 + c_{2,1} x_1^2 x_2^1
        \end{equation*}
        Nach konkreter Definition der Koeffizienten ergibt sich beispielsweise:
        \begin{equation*}
            R_{Bsp}(x_1, x_2) = 7 - \sqrt{\pi} x_2^1 + \frac{e}{\sqrt[3]{\gamma}} x_1^2 x_2^1
        \end{equation*}
    
    \subsubsection{Der Inhalt eines Polynoms}
        \label{subsubsec:content-poly}
        Sei R ein Polynom nach \hyperref[subsubsec:def-poly]{obiger Definition}.\ Dann sei der \emph{Inhalt} eines
        Polynoms wie folgt definiert:
        \begin{equation}
            \content{R} \coloneq \max\{|c_{\enum{j_1}{j_m}}| \} \  \forall \  0 \leq j_\mu \leq r_\mu, \
            1 \leq \mu \leq m \label{eq:def-content}
        \end{equation}
    
    \subsubsection{Restpolynom}
        \label{subsubsec:def-rempoly}
        Sei R ein Polynom nach \hyperref[subsubsec:def-poly]{obiger Definition} und seien $\enum{i_1}{i_m} \in
        \mathbb{N}_0$.\ Es wird
        \begin{equation}
            R_{\enum{i_1}{i_m}} = \den{\enum{i_1!}{i_m!}} \frac{\partial^{\dsum{i_1}{i_m}}}{
                \spdots{\partial x_1^{i_1}}{\partial x_m^{i_m}}} (R) \label{eq:def-rempoly}
        \end{equation}
        das \emph{Restpolynom} von R genannt.
    
    \subsubsection{Index eines Polynoms}
        \label{subsubsec:def-index}
        \textrm{Sei R ein Polynom nach \hyperref[subsubsec:def-poly]{obiger Definition}, $\alpha \in \mathbb{R}^m$
            und $s \in \mathbb{N}^m$.\ $I$ heißt \emph{Index} von R an der Stelle $\alpha$ bezüglich $s$, genau dann,
            wenn gilt:}
        \begin{equation}
            I \coloneq \ind(R) \coloneq \min_{(\enum{i_1}{i_m}) \in \mathbb{N}_0^m} \sum_{1 \leq \mu \leq m} \frac{i_\mu}
            {s_\mu}, \sptext{sodass} R_{i_1, \dots, i_m}(\alpha) \neq 0. \label{eq:def-index}
        \end{equation}
        Falls R verschwindet, wird konventionell $\ind(R) \coloneq \infty$ gesetzt.
        \newline
        \textrm{Dies erinnert an die \emph{Nullstellenordnung} einer Funktion (beispielsweise Nullstellen
        \emph{zweiter Ordnung} beziehungsweise \emph{zweifache} Nullstellen), lässt sich jedoch mit Hilfe des
        Vektors $s$ noch zusätzlich gewichten.
        \newline
        Zu diesem Begriff folgt wie erwähnt ein \hyperref[subsec:lemma2]{Lemma}, das weitere Eigenschaften
        darlegt.\ In dieser Arbeit wird diese Definition darüber hinaus jedoch nicht weiter verwendet; er findet lediglich
        in den Beweisen der weiteren Theorems im Beweis des \emph{Satzes von Roth} weitere Anwendung.}
    
